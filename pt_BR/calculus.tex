% !TeX root = main.tex
\clearpage\part{Cálculo}
\section{Introdução}

\subsection{O problema da área}
Há 2500 anos, os gregos achavam áreas através do método da exaustão, dividindo um polígono em triângulos para achar a área $A$ (Figura~\ref{fig:polygon-area-division}). Mas e a área de uma figura curva?  Figura~\ref{fig:circle-polygon-approximation} mostra que aumentando o número de lados do polígono inscrito, sua área se aproxima da área do círculo. Podemos utilizar esta técnica para encontrar a área de qualquer outra curva (Figura~\ref{fig:curve-approximation}). Este é um problema central para o \emph{cálculo integral}.\vspace{-0.6cm}
\begin{figure}[!ht]
  \subfloat[\label{fig:polygon-area-division}]{%
    \includegraphics{calculus/polygon-area-division}  
  }\vspace{-0.3cm}
  \subfloat[\label{fig:circle-polygon-approximation}]{%
    \includegraphics{calculus/circle-polygon-approximation}
  }\\
  \subfloat[\label{fig:curve-approximation}]{%
    \includegraphics{calculus/curve-approximation}
  }
  
  \caption{(a) $A = A_1+A_2+A_3+A_4+A_5$, (b) Encontrando a área do círculo através da aproximação do polígono, (c) Encontrar a área de uma curva através da aproximação do retângulo.}
\end{figure}

\subsection{O problema da tangente}
Considere o problema de encontrar uma equação da linha tangente $t$ a uma curva com equação $y=f(x)$ em um ponto $P$ (Figura~\ref{fig:tangent-line}). Sabendo que $P$ incide sobre a tangente, basta encontrar a inclinação $m$. O problema é que precisamos de dois pontos para encontrar $m$ e só temos o ponto $P$ em $t$. Podemos encontrar uma aproximação obtendo um ponto próximo $Q$ da curva e calcular a inclinação $m_{PQ}$ da secante $PQ$. Da Figura~\ref{fig:secant-line}:
\begin{equation}\label{eq:secant-slope}
m_{PQ} =\frac{f(x)-f(a)}{x-a}
\end{equation}
Imagine $Q$ se movendo sobre a curva em direção a $P$ (Figura~\ref{fig:secant-line-tangent}). A secante se rotaciona e se aproxima da tangente como sua posição limite. Isto significa que o valor da inclinação $m_{PQ}$ se torna próximo da inclinação $m$ da tangente, ou seja $m=\lim_{Q\rightarrow P}m_{PQ}$. Como $x$ se aproxima de $a$, pela Eq.~\ref{eq:secant-slope}:\begin{equation}\label{eq:tangent-slope}
m=\lim_{x\rightarrow a}\frac{f(x)-f(a)}{x-a}
\end{equation}
\vspace{-0.6cm}\begin{figure}[!ht]
  \subfloat[\label{fig:tangent-line}]{\includegraphics{calculus/tangent-line}}%
  \subfloat[\label{fig:secant-line}]{\includegraphics{calculus/secant-line}}%
  \hspace{-0.3cm}\subfloat[\label{fig:secant-line-tangent}]{\includegraphics{calculus/secant-line-tangent}}
  \caption{Problema da tangente: (a) tangente, (b) secante, (c) secante se aproximando da tangente.}
\end{figure}

O problema da tangente é fundamental para o \emph{cálculo diferencial}, inventado apenas 2000 anos após cálculo integral. As principais ideias por trás do cálculo diferencial foram do francês Pierre Fermat (1601-1665), desenvolvidas pelo inglês John Wallis (1616-1703), Isasc Barrow (1630-1677) e Isaac Newton (1642-1727), e pelo alemão Gottfried Leibniz (1646-1716). Há uma conexão forte entre esses dois problemas.

\subsection{Velocidade}
O que significa o velocímetro indicando 60km/h ? Sabemos que se a velocidade for constante, após uma hora viajaremos 60km. E se a velocidade variar? O que significa 60km/h em um dado instante? Vamos medir a distância a cada 1 segundo num exemplo:
\begin{table}[!ht]
  \centering
  \begin{tabular}{|>{\columncolor{bookbluearea}}l|c|c|c|c|c|c|}\hline
    t=Tempo gasto(s)&0&1&2&3&4&5\\\hline
    d=Distância(m)&0&2&9&24&42&71\\\hline
  \end{tabular}
\end{table}

\noindent Qual a velocidade média quando $2\leq t \leq 4$? $$\text{velocidade média} = \frac{\text{distância}}{\text{tempo}}=\frac{42-9}{4-2}=16,5 \text{m}/\text{s}$$
Qual a velocidade média quando $2\leq t \leq 3$?
$$\text{velocidade média} = \frac{\text{distância}}{\text{tempo}}=\frac{24-9}{3-2}=15 \text{m}/\text{s}$$
Temos a sensação que a velocidade instantânea em $t=2$ não é tão diferente da velocidade média em intervalos curtos começando por $t=2$. Que tal medir a cada 0.1 segundo?
\begin{table}[!ht]
  \centering
  \begin{tabular}{|>{\columncolor{bookbluearea}}l|c|c|c|c|c|c|}\hline
    t&2,0&2,1&2,2&2,3&2,4&2,5\\\hline
    d&9,00&10,02&11,16&12,45&13,96&15,80\\\hline
  \end{tabular}
\end{table}

\noindent Fazendo a mesma divisão a cada intervalo menor, teremos:
\begin{table}[!ht]
  \centering
  \setlength\tabcolsep{0.15cm}
  \begin{tabular}{|>{\columncolor{bookbluearea}}p{1.45cm}|c|c|c|c|c|c|}\hline
    \footnotesize Intervalo de tempo    &[2-3]&[2-2,5]&[2-2,4]&[2-2,3]&[2-2,2]&[2-2,1]\\\hline
    \footnotesize Velocidade média (m/s)&15,0&13,6&12,4&11,5&10,8&10,2\\\hline
  \end{tabular}
\end{table}

\noindent Reduzindo o intervalo, a velocidade se aproxima de 10. Assim, esperamos que a velocidade instantânea em $t=2$ seja próximo de 10 m/s. Figura~\ref{fig:circle-polygon-approximation} mostra o deslocamento do carro em função do tempo. A velocidade média no intervalo [2,$t$] é $$\text{velocidade média} = \frac{\text{distância}}{\text{tempo}}=\frac{f(t)-f(2)}{t-2}$$
que é igual à secante da Figura~\ref{fig:circle-polygon-approximation}. A velocidade instantânea $v$ quando $t=2$ é o valor limite desta velocidade média quanto $t$ se aproxima de 2, $$v=\lim_{t\rightarrow 2}\frac{f(t)-f(2)}{t-2}$$ e da Eq.~\ref{eq:tangent-slope} vemos que $v$ é a inclinação da tangente da curva em $P$.
\begin{figure}[!ht]
	\centering
  \includegraphics{calculus/velocity_curve}
  \caption{Problema da tangente: (a) tangente, (b) secante, (c) secante se aproximando da tangente.}
\end{figure}

Podemos aplicar as mesmas técnicas de tangentes em cálculo diferencial não só em velocidades mas em todas as ciências sociais e naturais.

\subsection{O limite de uma sequência}
No 5º século A.C. o filósofo grego Zenão de Eléia propôs 4 problemas, conhecidos como \emph{paradoxos de Zenão}, desafiando ideias de tempo e espaço na época. O 2º paradoxo diz respeito a uma corrida entre o herói grego Aquiles e uma tartaruga com vantagem inicial. Zenão argumentou que Aquiles nunca passaria a tartaruga: suponha que Aquiles inicie na posição $a_1$ e a tartaruga em $t_1$ (Figura~\ref{fig:circle-polygon-approximation}). Quando Aquiles chega em $a_2 = t_1$, a tartaruga está em $t_2$ e quando chega em $a_3=t_2$, a tartaruga está em $t_3$. Este processo continua indefinidamente, parecendo que a tartaruga sempre estará na frente, mesmo desafiando o senso comum:
\begin{figure}[!ht]
  \centering
  \begin{minipage}[b]{0.48\columnwidth}
  	\centering
  	\subfloat[\label{fig:achilles_tortoise}]
  	{\includegraphics[width=\textwidth]{calculus/achilles_tortoise}}
  	\vfill
  	\subfloat[\label{fig:sequence_numerical_line}]
  	{\includegraphics[width=\textwidth]{calculus/sequence_numerical_line}}
  \end{minipage}
	\sbox{\measurebox}{%
		\begin{minipage}[b][\ht\measurebox]{0.48\columnwidth}
			\subfloat[\label{fig:sequence_chart}]
			{\includegraphics[width=\columnwidth]{calculus/sequence_chart}}
	\end{minipage}}
  \usebox{\measurebox}\qquad
  \caption{(a) 2º paradoxo de Zenão, a corrida entre Aquiles e a tartaruga nunca alcançada, (b) sequência em uma linha numérica e (c) em um gráfico.-}
\end{figure}

\noindent Uma maneira de explicar este paradoxo é a ideia de \emph{sequência}. As posições sucessivas de Aquiles ($a_1$,$a_2$,$a_3$,$\dots$) ou da tartaruga ($t_1$,$t_2$,$t_3$,$\dots$) formam uma sequência. Em geral, uma sequência $\{a_n\}$ é um conjunto de números em ordem. Por exemplo, $\left\{1,\frac{1}{2},\frac{1}{3},\frac{1}{4},\frac{1}{5},\dots\right\}$ pode ser descrita pela expressão $a_n=\frac{1}{n}$. Podemos visualizar esta sequência plotando seus termos em uma linha numérica (Figura~\ref{fig:sequence_numerical_line}) ou num gráfico (Figura~\ref{fig:sequence_chart}). Observe que os termos se tornam mais próximos de 0 quando $n$ aumenta. Dizemos que o limite da sequência é 0, ou seja $$\lim_{n\rightarrow\infty}\frac{1}{n}=0$$
Em geral, a notação $\lim_{n\rightarrow\infty}a_n=L$ é usada se os termos $a_n$ se aproximam do número $L$ quando $n$ aumenta.
O conceito de limite de sequência ocorre toda vez que usamos a representação decimal de um número real. Por exemplo:
\begin{table}[!ht]
  \centering
  \setlength\tabcolsep{0.15cm}
  \begin{tabular}{|c|c|c|c|c|c|c|}\hline
    $a_1$&$a_2$&$a_3$&$a_4$&$a_5$&$a_6$&$a_7$\\\hline
    3,1&3,14&3,141&3,1415&3,14159&3,141592&3,1415926\\\hline
  \end{tabular}
\end{table}

\noindent Então os termos desta sequência são aproximações racionais de $\pi$, ou seja, $\lim_{n\rightarrow \infty}a_n=\pi$. Voltando ao paradoxo de Zenão. As posições de Aquiles e da tartaruga formam sequências $\{a_n\}$ e $\{t_n\}$, onde $a_n<t_n$, para todo $n$. Pode-se mostrar que ambas as sequências têm o mesmo limite: $\lim_{n\rightarrow\infty}a_n=p=\lim_{n\rightarrow\infty}t_n$. É neste ponto $p$ onde Aquiles supera a tartaruga.

\subsection{Soma de uma série}
Outro paradoxo de Zenão, dito por Aristóteles, é o seguinte: "Um homem andando no meio da sala não conseguirá alcançar a parede". Para isso, ele passaria pela metade da distância, sobrando a outra metade. Isso aconteceria indefinidamente (Figura~\ref{fig:achilles_tortoise}).
\begin{figure}[!ht]
	\centering
	\includegraphics{calculus/wall}
	\caption{Outro paradoxo de Zenão: o homem nunca alcança a parede.}
\end{figure}
Sabendo que o homem vai alcançar a parede, podemos expressar a distância total: $1=\frac{1}{2}+\frac{1}{4}+\frac{1}{8}+\frac{1}{16}+\dots+\frac{1}{2^n}+\dots$. Zenão argumentava que não fazia sentido adicionar infinitamente muitos números, porém há outros casos mais claros de somas infinitas. Por exemplo, em notação decimal, o número $0,\overline{3} = 0,3333\dots$ é $\frac{3}{10}+\frac{3}{100}+\frac{3}{1000}+\dots=\frac{1}{3}$. De modo geral, se $d_n$ denota o $n$-ésimo dígito na representação decimal de um número, então $$0,d_1d_2d_3\dots = \frac{d_1}{10}+\frac{d_2}{10^2}+\frac{d_3}{10^3}+\dots+\frac{d_n}{10^n}+\dots$$
Voltando ao paradoxo, definindo os termos da série como:$$\begin{aligned}
s_1 &= \frac{1}{2}= 0,5\\
s_2 &= \frac{1}{2}+\frac{1}{4} = 0,75\\
s_3 &= \frac{1}{2}+\frac{1}{4}+\frac{1}{8} = 0,875\\
s_4 &= \frac{1}{2}+\frac{1}{4}+\frac{1}{8}+\frac{1}{16}=0.9375\\
%s_5 &= \frac{1}{2}+\frac{1}{4}+\frac{1}{8}+\frac{1}{16}+\frac{1}{32}=0.96875\\
%s_6 &= \frac{1}{2}+\frac{1}{4}+\frac{1}{8}+\frac{1}{16}+\frac{1}{32}+\frac{1}{64}=0.984375\\
\vdots\\
s_{16} &= \frac{1}{2}+\frac{1}{4}+\dots+\frac{1}{2^{16}}\approx 0,99998474
\end{aligned}$$
A soma parcial se torna próxima de 1. Aumentando $n$, $s_n$ se torna mais próximo de 1. Torna-se razoável definir a soma da série infinita como 1: $$\frac{1}{2}+\frac{1}{4}+\dots+\frac{1}{2^n} = 1$$ A razão por ser 1 é: $\lim_{n\rightarrow\infty}s_n=1$. Usaremos as ideias de Newton para combinar séries infinitas com cálculo diferencial e integral. 
\subsection{Sumário}
Vemos que o conceito de limites aparece ao tentar achar a área de uma região, a inclinação de uma tangente a uma curva, a velocidade do carro ou a soma de uma série infinita. O tema em comum é o cálculo de um número que é limite de outro calculado mais facilmente. Este conceito de limite diferencia o cálculo das outras áreas da matemática. Na verdade, podemos definir cálculo como a área da matemática que lida com limites. Após Sir Isaac Newton ter inventado sua versão do cálculo, ele o usou para explicar o movimento dos planetas em torno do sol. Hoje o cálculo é usado para calcular as órbitas dos satélites e espaçonaves, predizer tamanhos de populações, estimar taxa de variação do preço do petróleo, prever mudanças climáticas, medir batimento cardíaco, calcular seguros, e muitas outras áreas. As seguintes perguntas nos dizem o poder desta disciplina:
\begin{enumerate}
  \item Como podemos explicar que o ângulo de elevação de um observador até o topo do arco-íris é $42\degree$?
  \item Como podemos explicar o formato das latas nas prateleiras de um supermercado?
  \item Qual é o melhor lugar para se sentar em um cinema?
  \item Como podemos projetar uma montanha russa de forma que seja totalmente suave?
  \item O quão longe do aeroporto o piloto deve começar o pouso?
  \item Como unir curvas para formar letras numa impressora a laser?
  \item Como estimar o número de trabalhadores necessários para construir a Grande Pirâmide de Khufu no Egito Antigo?
  \item Como podemos explicar o fato dos planetas e satélites se moverem em órbitas elípticas?
  \item Como distribuir o fluxo de água nas turbinas de uma estação hidroelétrica para maximizar a produção de energia?
\end{enumerate}

\section{Funções e limites}
Funções aparecem quando quantidades dependem de outras. Considere essas 4 situações:
\begin{itemize}
  \item A área $A$ de um círculo depende do raio $r$. A regra que conecta $r$ e $A$ é dada por: $A=\pi r^2$. Associa-se cada valor positivo de $r$ a um valor de $A$. Assim, $A$ é uma \emph{função} de $r$.
  \item A população mundial $P$ depende do tempo $t$. A Tabela abaixo mostra estimativas de $P(t)$ no tempo $t$, para alguns anos. Por exemplo, $P(1950)\approx 2.560.000.000$. Para cada valor de $t$ há um correspondente de $P$ e $P$ é em função de $t$.\begin{table}[!ht]
    \centering
    \vspace{-0.25cm}
    \setlength\tabcolsep{0.1cm}
    \begin{tabular}{|>{\centering\columncolor{bookbluearea}}m{1.5cm}|c|c|c|c|c|c|c|c|}\hline
      Ano&1940&1950&1960&1970&1980&1990&2000&2010\\\hline
      População (milhões)&2300&2560&3040&3710&4450&5280&6080&6870\\\hline
    \end{tabular}
    \vspace{-0.25cm}
  \end{table}
  \item O custo $C$ de enviar um envelope depende do peso $p$. Embora não seja simples conectar $p$ e $C$, os correios definem uma regra para determinar $C$ quando $p$ é conhecido.
  \item A aceleração vertical $a$ do terreno medido por um sismógrafo durante um terremoto é uma função do tempo $t$. Figura~\ref{fig:earthquake} mostra um gráfico da atividade sísmica durante o terremoto de Northidge que abalou Los Angeles em 1994. Para um dado valor de $t$, o gráfico mostra o valor de $a$.\vspace{-0.2cm}\begin{figure}[!ht]
    \centering
    \includegraphics{calculus/northridge-earthquake}
    \caption{Aceleração vertical do terreno durante o terremoto de Northridge.}
    \label{fig:earthquake}
  \end{figure}
\end{itemize}
Em todos os casos, dizemos que o segundo valor ($A$, $P$, $C$ ou $a$) é uma função do primeiro ($r$,$t$,$p$ ou $t$). Uma \textbf{função} $f$ é uma regra que atribui a cada elemento $x$ em um conjunto $D$ exatamente um elemento, chamado $f(x)$, em um conjunto $E$. De forma compacta, $f:D\rightarrow E$. Geralmente $D=\mathds{R}$ e $E=\mathds{R}$ (números reais). $D$ é o \textbf{domínio} da função e $E$ é o \textbf{contradomínio}. $f(x)$ é o \textbf{valor de $f$ em $x$} e se lê "$f$ de $x$". O símbolo que representa um número no domínio é a \textbf{variável independente}. O símbolo que representa um número na imagem é a \textbf{variável independente}.

Uma função também pode ser interpretada como uma \textbf{máquina} (Figura~\ref{fig:machine-diagram}), aceitando um elemento $x$ do domínio da função $f$ na entrada e devolvendo a imagem $f(x)$ na saída. Um exemplo são as funções da calculadora.
\begin{figure}[!ht]
  \centering
  \vspace{0.0cm}
  \begin{minipage}[b]{0.3\columnwidth}
    \centering
    \subfloat[\label{fig:machine-diagram}]{\includegraphics[width=\columnwidth]{calculus/machine-diagram}}%
    \vfill
    \subfloat[\label{fig:arrow-diagram}]{\includegraphics[width=\columnwidth]{calculus/arrow-diagram}}%
  \end{minipage}
  \sbox{\measurebox}{%
    \begin{minipage}[b][\ht\measurebox]{0.68\columnwidth}
      \subfloat[\label{fig:graph}]{\includegraphics[width=0.54\columnwidth]{calculus/graph}}\hspace{0.01cm}
      \subfloat[\label{fig:graph2}]{\includegraphics[width=0.41\columnwidth]{calculus/graph2}}
  \end{minipage}}
  \usebox{\measurebox}\qquad
  \caption{Função como: (a) máquina, (b) diagrama de setas e (c) gráfico. (d) Domínio e imagem de uma função.}
\end{figure}

Outra interpretação de $f$ é como um \textbf{diagrama de setas} (Figura~\ref{fig:arrow-diagram}), conectando domínio $D$ e imagem $E$, associando $a$ e $f(a)$, $x$ e $f(x)$, etc. Por fim, a interpretação mais comum é como um \textbf{gráfico} (Figura~\ref{fig:graph}), mostrando pares ordenados $\left\{(x,f(x)):x\in D\right\}$ como pontos (x,y) num plano de coordenadas, ilustrando a "história de vida" de $f$ e podendo indicar também o domínio e a imagem (Figura~\ref{fig:graph2}).

\example{1} Seja $f$ do gráfico da Figura~\ref{fig:graph}.
\begin{enumerate}[label=(\alph*)]
  \item Encontre os valores de $f(1)$ e $f(5)$.
  \item Qual é o domínio e imagem de $f$?
\end{enumerate}
\solution
\begin{enumerate}[label=(\alph*)]
  \item Vemos que o ponto (1, 3) está no gráfico de $f$, então $f(1) = 3$. Contando 5 unidades no eixo $x$, $f(5)$ se situa cerca de 0.7 unidades abaixo do eixo $y$ ($f(5)\approx-0.7$).
  \item $f(x)$ é definido quando $0\leq x \leq 7$, então o domínio é o intervalo [0,7]. Veja que f é definido quando $-2\leq y \leq 4$, então a imagem é o intervalo [-2,4].
\end{enumerate}

\example{2} Desenhe o gráfico e ache o domínio e imagem de:
\vspace{-0.7cm}\begin{multicols}{2}
  \begin{enumerate}[label=(\alph*)]
    \item $f(x)=2x-1$
    \item $g(x)=x^2$
  \end{enumerate}
\end{multicols}\vspace{-0.3cm}
\solution
\begin{enumerate}[label=(\alph*)]
  \item A equação do gráfico é $y=2x-1$, uma linha de inclinação 2 e intercepto -1. Figura~\ref{fig:ex2-line} mostra o gráfico de $f$. A expressão $2x-1$ é definida para todos os números reais, então o domínio de $f$ é $\mathds{R}$. Veja que a imagem também é $\mathds{R}$.
  \item Como $g(2)= 2^2 = 4$ e $g(-1) = (-1)^2 = 1$, podemos plotar os pontos $(2,4)$ e $(-1,1)$, e uni-los com outros pontos no gráfico (Figura~\ref{fig:ex2-parabola}). A equação do gráfico é $y=x^2$, que representa uma parábola. O domínio de $g$ é $\mathds{R}$. A imagem de $g$ consiste em todos os valores de $g(x)$, ou seja, todos os números na forma $x^2$. Mas $x^2\geq 0, \forall x$. Assim, a imagem de $g$ é $\{y:y\geq 0\} = [0,\infty)$.\vspace{-0.3cm}
\end{enumerate}
\begin{figure}[!ht]
  \centering
  \subfloat[\label{fig:ex2-line}]{\includegraphics{calculus/ex2-line}}%
  \subfloat[\label{fig:ex2-parabola}]{\includegraphics{calculus/ex2-parabola}}%
  \caption{Gráficos do EXEMPLO 2}
\end{figure}\vspace{-0.4cm}
\example{3} Se $f(x)=2x^2-5x+1$ e $h\neq 0$, avalie $\displaystyle\frac{f(a+h)-f(a)}{h}$ o qual é denominada \textbf{quociente diferencial}, a taxa de variação de f(x) entre $x=a$ e $x=a+h$

\solution

\noindent Obtemos $f(a+h)$ trocando $x$ por $a+h$ na equação de $f(x)$:
$$
\begin{aligned}
f(a+h) &=2(a+h)^2-5(a+h)+1\\
&= 2(a^2+2ah+h^2)-5(a+h)+1\\
&=2a^2+4ah+2h^2-5a-5h+1
\end{aligned}
$$
Então substituímos na expressão dada e o simplificamos: $$
\begin{aligned}
\frac{f(a+h)-f(a)}{h} &= \frac{(2a^2+4ah+2h^2-5a-5h+1)-(2a^2-5a+1)}{h}\\
&=\frac{2a^2+4ah+2h^2-5a-5h+1-2a^2+5a-1}{h}\\
&=\frac{4ah+2h^2-5h}{h}=4a+2h-5
\end{aligned}
$$
\subsection{Representação de funções}
Há 4 formas de representar uma função: verbalmente (por uma descrição em palavras), numericamente (por uma tabela de valores), visualmente (por um gráfico) e algebricamente (por uma fórmula explícita). É interessante saber como mudar representações de uma dada função (No Exemplo 2 iniciamos com fórmulas algébricas e obtivemos os gráficos).

A representação mais útil da área de um círculo como função do raio é provavelmente a fórmula algébrica $A(r)=\pi r^2$, embora seja possível compilar uma tabela de valores ou desenhar um gráfico (metade uma parábola). O domínio é $\{r:r>0\}=(0,\infty)$ e a imagem também é $(0,\infty)$.

Um exemplo de descrição de uma função verbalmente seria: $P(t)$ é a população humana mundial no tempo $t$. Seja $t=0$ correspondendo ao ano 1900. Podemos gerar uma tabela de valores e logo em seguida um gráfico de dispersão (Figura~\ref{fig:population-scatter}), que nos ajuda a entender todos os dados de uma vez. Falta só a fórmula que descreva de forma exata a população, o que pode ser impossível. Mas podemos encontrar uma fórmula que se aproxima de $P(t)$: $P(t) \approx f(t) = (1,43653\times 10^9)\cdot(1,01395)^t$. Figura~\ref{fig:population-scatter-fit} mostra que f(t) é um bom "ajuste". A função $f$ é um \emph{modelo matemático} para o crescimento populacional. Em outras palavras, é uma função com uma fórmula explícita que aproxima o comportamento de uma dada função. Contudo, veremos que as ideias do cálculo podem ser aplicadas a uma tabela de valores; uma fórmula explícita não é necessária.
\begin{figure}[!ht]
	\centering
	\subfloat[\label{fig:population-scatter}]{\includegraphics{calculus/ex2-line}}%
	\subfloat[\label{fig:population-scatter-fit}]{\includegraphics{calculus/ex2-parabola}}%
	\caption{Gráficos do EXEMPLO 2}
\end{figure}

A função $P$ é típica de funções que aparecem no mundo real. Começamos com uma descrição verbal (uma hipótese), depois realizamos observações e medições utilizando aparatos e métodos científicos, criando uma tabela de valores. Em seguida podemos derivar uma função aproximada ou aplicar operações de cálculo, mesmo que não tenhamos o total conhecimento do comportamento da função.

Outro exemplo descrito em palavras: seja $C(p)$ o custo de enviar um envelope de peso $p$. A regra dos Correios para Carta Comercial (preço básico) é: 
\begin{table}
	\begin{tabular}{lccccc}
		Peso & $p \leq 20$ & $20 > p \leq 50$ & $50 > p \leq 100$ & $100 > p \leq 150$ & $20 > p \leq 50$\\
		Preço & $p \leq 20$ & $20 > p \leq 50$ & $50 > p \leq 100$ & $100 > p \leq 150$ & $20 > p \leq 50$\\
	\end{tabular}
\end{table}
Até 20	1,80	6,10	10,40	12,00	16,30
Mais de 20 até   50	2,55	6,85	11,15	12,75	17,05
Mais de 50 até 100	3,50	7,80	12,10	13,70	18,00
Mais de 100 até 150	4,25	8,55	12,85	14,45	18,75
Mais de 150 até 200	5,05	9,35	13,65	15,25	19,55
Mais de 200 até 250	5,85	10,15	14,45	16,05	20,35
Mais de 250 até 300	6,65	10,95	15,25	16,85	21,15
Mais de 300 até 350	7,45	11,75	16,05	17,65	21,95
Mais de 350 até 400	8,20	12,50	16,80	18,40	22,70
Mais de 400 até 450	9,00	13,30	17,60	19,20	23,50
Mais de 450 até 500	9,80	14,10	18,40	20,00	24,30

\subsection{Teste da linha vertical}

A curva no plano $xy$ é o gráfico de uma função se e somente se nenhuma linha vertical intersecta a curva mais de uma vez. O caso contrário significa que a função mapeia um elemento do domínio para mais de um elemento da imagem, o que não pode acontecer. Por exemplo, a parábola $x = y^2-2$ da Figura~\ref{fig:achilles_tortoise} não é o gráfico de uma função pois há linhas verticais intersectando a parábola duas vezes. Na verdade, tal parábola é um gráfico de duas funções. Veja que $y^2=x+2$ e $y=\pm\sqrt{x+2}$, então a metade superior representa $f(x) = \sqrt{x+2}$; a inferior, $g(x)=-\sqrt{x+2}$. Revertendo os papeis de x e y (x sendo uma função h(y)), a parábola original se torna o gráfico da função h(y).
\begin{figure}[!ht]
	\centering
	\subfloat[\label{fig:parabola-complete}]{\includegraphics{calculus/ex2-line}}%
	\subfloat[\label{fig:parabola-half}]{\includegraphics{calculus/ex2-parabola}}%
	\subfloat[\label{fig:parabola-other-halp}]{\includegraphics{calculus/ex2-parabola}}%
	\caption{Gráficos do EXEMPLO 2}
\end{figure}

\example{7} A função $f$ é definida como $$f(x) = \begin{cases}
1-x, & \text{ se } x \leq -1\\
x^2, & \text{ se } x > -1\\
\end{cases}$$
Avalie $f(-2)$, $f(-1)$ e $f(0)$ e desenhe o gráfico.
\solution
Mesmo tendo duas regras diferentes, estamos tratando de uma só função, pois elas se referem a faixas diferentes no domínio (não se sobrepõem). A função devolve o valor de $y$ usando a primeira regra apenas se $x \leq 1$, aplicando a segunda regra caso contrário ($x > 1$).

\section{Derivadas}
\section{Integrais}
\section{Funções inversas}
\section{Técnicas de integração}
\section{Aplicações de integração}
\section{Equações diferenciais}
\section{Equações paramétricas e coordenadas polares}
\section{Sequências infinitas e séries}
