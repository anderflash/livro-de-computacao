% !TeX root = main.tex
\clearpage\part{Cálculo}
\section{Introdução}

\subsection{O problema da área}
Há 2500 anos, os gregos achavam áreas através do método da exaustão, dividindo um polígono em triângulos para achar a área $A$ (Figura~\ref{fig:polygon-area-division}). Mas e a área de uma figura curva?  Figura~\ref{fig:circle-polygon-approximation} mostra que aumentando o número de lados do polígono inscrito, sua área se aproxima da área do círculo. Podemos utilizar esta técnica para encontrar a área de qualquer outra curva (Figura~\ref{fig:curve-approximation}). Este é um problema central para o \emph{cálculo integral}.
\vspace{-0.6cm}
\begin{figure}[!ht]
  \subfloat[\label{fig:polygon-area-division}]{%
    \includegraphics{calculus/polygon-area-division}  
  }\vspace{-0.3cm}
  \subfloat[\label{fig:circle-polygon-approximation}]{%
    \includegraphics{calculus/circle-polygon-approximation}
  }\\
  \subfloat[\label{fig:curve-approximation}]{%
    \includegraphics{calculus/curve-approximation}
  }
  
  \caption{(a) $A = A_1+A_2+A_3+A_4+A_5$, (b) Encontrando a área do círculo através da aproximação do polígono, (c) Encontrar a área de uma curva através da aproximação do retângulo.}
\end{figure}

\subsection{O problema da tangente}
Considere o problema de encontrar uma equação da linha tangente $t$ a uma curva com equação $y=f(x)$ em um ponto $P$ (Figura~\ref{fig:tangent-line}). Sabendo que $P$ incide sobre a tangente, basta encontrar o declive $m$. O problema é que precisamos de dois pontos para encontrar $m$ e só temos o ponto $P$ em $t$. Podemos encontrar uma aproximação obtendo um ponto próximo $Q$ da curva e calcular o declive $m_{PQ}$ da secante $PQ$. Da Figura~\ref{fig:secant-line}:
\begin{equation}\label{eq:secant-slope}
m_{PQ} =\frac{f(x)-f(a)}{x-a}
\end{equation}
Imagine $Q$ se movendo sobre a curva em direção a $P$ (Figura~\ref{fig:secant-line-tangent}). A secante se rotaciona e se aproxima da tangente como sua posição limite. Isto significa que o valor do declive $m_{PQ}$ se torna próximo do declive $m$ da tangente, ou seja $m=\lim_{Q\rightarrow P}m_{PQ}$. Como $x$ se aproxima de $a$, pela Eq.~\ref{eq:secant-slope}:\begin{equation}\label{eq:tangent-slope}
m=\lim_{x\rightarrow a}\frac{f(x)-f(a)}{x-a}
\end{equation}
\vspace{-0.6cm}\begin{figure}[!ht]
  \subfloat[\label{fig:tangent-line}]{\includegraphics{calculus/tangent-line}}%
  \subfloat[\label{fig:secant-line}]{\includegraphics{calculus/secant-line}}%
  \hspace{-0.3cm}\subfloat[\label{fig:secant-line-tangent}]{\includegraphics{calculus/secant-line-tangent}}
  \caption{Problema da tangente: (a) tangente, (b) secante, (c) secante se aproximando da tangente.}
\end{figure}

O problema da tangente é fundamental para o \emph{cálculo diferencial}, inventado apenas 2000 anos após cálculo integral. As principais ideias por trás do cálculo diferencial foram do francês Pierre Fermat (1601-1665), desenvolvidas pelo inglês John Wallis (1616-1703), Isasc Barrow (1630-1677) e Isaac Newton (1642-1727), e pelo alemão Gottfried Leibniz (1646-1716). Há uma conexão forte entre esses dois problemas.

\subsection{Velocidade}
O que significa o velocímetro indicando 60km/h ? Sabemos que se a velocidade for constante, após uma hora viajaremos 60km. E se a velocidade variar? O que significa 60km/h em um dado instante? Vamos medir a distância a cada 1 segundo num exemplo:
\begin{table}[!ht]
  \centering
  \begin{tabular}{|>{\columncolor{bookbluearea}}l|c|c|c|c|c|c|}\hline
    t=Tempo gasto(s)&0&1&2&3&4&5\\\hline
    d=Distância(m)&0&2&9&24&42&71\\\hline
  \end{tabular}
\end{table}

\noindent Qual a velocidade média quando $2\leq t \leq 4$? $$\text{velocidade média} = \frac{\text{distância}}{\text{tempo}}=\frac{42-9}{4-2}=16,5 \text{m}/\text{s}$$
Qual a velocidade média quando $2\leq t \leq 3$?
$$\text{velocidade média} = \frac{\text{distância}}{\text{tempo}}=\frac{24-9}{3-2}=15 \text{m}/\text{s}$$
Temos a sensação que a velocidade instantânea em $t=2$ não é tão diferente da velocidade média em intervalos curtos começando por $t=2$. Que tal medir a cada 0.1 segundo?
\begin{table}[!ht]
  \centering
  \begin{tabular}{|>{\columncolor{bookbluearea}}l|c|c|c|c|c|c|}\hline
    t&2,0&2,1&2,2&2,3&2,4&2,5\\\hline
    d&9,00&10,02&11,16&12,45&13,96&15,80\\\hline
  \end{tabular}
\end{table}

\noindent Fazendo a mesma divisão a cada intervalo menor, teremos:
\begin{table}[!ht]
  \centering
  \setlength\tabcolsep{0.15cm}
  \begin{tabular}{|>{\columncolor{bookbluearea}}p{1.45cm}|c|c|c|c|c|c|}\hline
    \footnotesize Intervalo de tempo    &[2-3]&[2-2,5]&[2-2,4]&[2-2,3]&[2-2,2]&[2-2,1]\\\hline
    \footnotesize Velocidade média (m/s)&15,0&13,6&12,4&11,5&10,8&10,2\\\hline
  \end{tabular}
\end{table}

\noindent Reduzindo o intervalo, a velocidade se aproxima de 10. Assim, esperamos que a velocidade instantânea em $t=2$ seja próximo de 10 m/s. Figura~\ref{fig:circle-polygon-approximation} mostra o deslocamento do carro em função do tempo. A velocidade média no intervalo [2,$t$] é $$\text{velocidade média} = \frac{\text{distância}}{\text{tempo}}=\frac{f(t)-f(2)}{t-2}$$
que é igual à secante da Figura~\ref{fig:circle-polygon-approximation}. A velocidade instantânea $v$ quando $t=2$ é o valor limite desta velocidade média quanto $t$ se aproxima de 2, $$v=\lim_{t\rightarrow 2}\frac{f(t)-f(2)}{t-2}$$ e da Eq.~\ref{eq:tangent-slope} vemos que $v$ é o declive da tangente da curva em $P$.
\begin{figure}[!ht]
	\centering
  \includegraphics{calculus/velocity_curve}
  \caption{Problema da tangente: (a) tangente, (b) secante, (c) secante se aproximando da tangente.}
\end{figure}

Podemos aplicar as mesmas técnicas de tangentes em cálculo diferencial não só em velocidades mas em todas as ciências sociais e naturais.

\subsection{O limite de uma sequência}
No 5º século A.C. o filósofo grego Zenão de Eléia propôs 4 problemas, conhecidos como \emph{paradoxos de Zenão}, desafiando ideias de tempo e espaço na época. O 2º paradoxo diz respeito a uma corrida entre o herói grego Aquiles e uma tartaruga com vantagem inicial. Zenão argumentou que Aquiles nunca passaria a tartaruga: suponha que Aquiles inicie na posição $a_1$ e a tartaruga em $t_1$ (Figura~\ref{fig:circle-polygon-approximation}). Quando Aquiles chega em $a_2 = t_1$, a tartaruga está em $t_2$ e quando chega em $a_3=t_2$, a tartaruga está em $t_3$. Este processo continua indefinidamente, parecendo que a tartaruga sempre estará na frente, mesmo desafiando o senso comum:
\begin{figure}[!ht]
  \vspace{-.4cm}
  \centering
  \begin{minipage}[b]{0.48\columnwidth}
  	\centering
  	\subfloat[\label{fig:achilles_tortoise}]
  	{\includegraphics[width=\textwidth]{calculus/achilles_tortoise}}
  	\vfill
  	\subfloat[\label{fig:sequence_numerical_line}]
  	{\includegraphics[width=\textwidth]{calculus/sequence_numerical_line}}
  \end{minipage}
	\sbox{\measurebox}{%
		\begin{minipage}[b][\ht\measurebox]{0.48\columnwidth}
			\subfloat[\label{fig:sequence_chart}]
			{\includegraphics[width=\columnwidth]{calculus/sequence_chart}}
	\end{minipage}}
  \usebox{\measurebox}\qquad
  \caption{(a) 2º paradoxo de Zenão, a corrida entre Aquiles e a tartaruga nunca alcançada, (b) sequência em uma linha numérica e (c) em um gráfico.-}\vspace{-0.2cm}
\end{figure}

\noindent Uma maneira de explicar este paradoxo é a ideia de \emph{sequência}. As posições sucessivas de Aquiles ($a_1$,$a_2$,$a_3$,$\dots$) ou da tartaruga ($t_1$,$t_2$,$t_3$,$\dots$) formam uma sequência. Em geral, uma sequência $\{a_n\}$ é um conjunto de números em ordem. Por exemplo, $\left\{1,\frac{1}{2},\frac{1}{3},\frac{1}{4},\frac{1}{5},\dots\right\}$ pode ser descrita pela expressão $a_n=\frac{1}{n}$. Podemos visualizar esta sequência plotando seus termos em uma linha numérica (Figura~\ref{fig:sequence_numerical_line}) ou num gráfico (Figura~\ref{fig:sequence_chart}). Observe que os termos se tornam mais próximos de 0 quando $n$ aumenta. Dizemos que o limite da sequência é 0, ou seja $$\lim_{n\rightarrow\infty}\frac{1}{n}=0$$
Em geral, a notação $\lim_{n\rightarrow\infty}a_n=L$ é usada se os termos $a_n$ se aproximam do número $L$ quando $n$ aumenta.
O conceito de limite de sequência ocorre toda vez que usamos a representação decimal de um número real. Por exemplo:
\begin{table}[!ht]
  \vspace{-0.3cm}
  \centering
  \setlength\tabcolsep{0.15cm}
  \begin{tabular}{|c|c|c|c|c|c|c|}\hline
    $a_1$&$a_2$&$a_3$&$a_4$&$a_5$&$a_6$&$a_7$\\\hline
    3,1&3,14&3,141&3,1415&3,14159&3,141592&3,1415926\\\hline
  \end{tabular}
  \vspace{-0.3cm}
\end{table}

\noindent Então os termos desta sequência são aproximações racionais de $\pi$, ou seja, $\lim_{n\rightarrow \infty}a_n=\pi$. Voltando ao paradoxo de Zenão. As posições de Aquiles e da tartaruga formam sequências $\{a_n\}$ e $\{t_n\}$, onde $a_n<t_n$, para todo $n$. Pode-se mostrar que ambas as sequências têm o mesmo limite: $\lim_{n\rightarrow\infty}a_n=p=\lim_{n\rightarrow\infty}t_n$. É neste ponto $p$ onde Aquiles supera a tartaruga.

\subsection{Soma de uma série}
Outro paradoxo de Zenão, dito por Aristóteles, é o seguinte: "Um homem andando no meio da sala não conseguirá alcançar a parede". Para isso, ele passaria pela metade da distância, sobrando a outra metade. Isso aconteceria indefinidamente (Figura~\ref{fig:achilles_tortoise}).
\begin{figure}[!ht]
  \vspace{-0.3cm}
	\centering
	\includegraphics{calculus/wall}
	\caption{Outro paradoxo de Zenão: nunca alcançar a parede.}
  \vspace{-0.3cm}
\end{figure}

Sabendo que o homem vai alcançar a parede, podemos expressar a distância total: $1=\frac{1}{2}+\frac{1}{4}+\frac{1}{8}+\frac{1}{16}+\dots+\frac{1}{2^n}+\dots$. Zenão argumentava que não fazia sentido adicionar infinitamente muitos números, porém há outros casos mais claros de somas infinitas. Por exemplo, em notação decimal, o número $0,\overline{3} = 0,3333\dots$ é $\frac{3}{10}+\frac{3}{100}+\frac{3}{1000}+\dots=\frac{1}{3}$. De modo geral, se $d_n$ denota o $n$-ésimo dígito na representação decimal de um número, então $$0,d_1d_2d_3\dots = \frac{d_1}{10}+\frac{d_2}{10^2}+\frac{d_3}{10^3}+\dots+\frac{d_n}{10^n}+\dots$$
Voltando ao paradoxo, definindo os termos da série como:$$\begin{aligned}
s_1 &= \frac{1}{2}= 0,5\\
s_2 &= \frac{1}{2}+\frac{1}{4} = 0,75\\
s_3 &= \frac{1}{2}+\frac{1}{4}+\frac{1}{8} = 0,875\\
s_4 &= \frac{1}{2}+\frac{1}{4}+\frac{1}{8}+\frac{1}{16}=0.9375\\
%s_5 &= \frac{1}{2}+\frac{1}{4}+\frac{1}{8}+\frac{1}{16}+\frac{1}{32}=0.96875\\
%s_6 &= \frac{1}{2}+\frac{1}{4}+\frac{1}{8}+\frac{1}{16}+\frac{1}{32}+\frac{1}{64}=0.984375\\
\vdots\\
s_{16} &= \frac{1}{2}+\frac{1}{4}+\dots+\frac{1}{2^{16}}\approx 0,99998474
\end{aligned}$$
A soma parcial se torna próxima de 1. Aumentando $n$, $s_n$ se torna mais próximo de 1. Torna-se razoável definir a soma da série infinita como 1: $$\frac{1}{2}+\frac{1}{4}+\dots+\frac{1}{2^n} = 1$$ A razão por ser 1 é: $\lim_{n\rightarrow\infty}s_n=1$. Usaremos as ideias de Newton para combinar séries infinitas com cálculo diferencial e integral. 
\subsection{Sumário}
Vemos que o conceito de limites aparece ao tentar achar a área de uma região, o declive de uma tangente a uma curva, a velocidade do carro ou a soma de uma série infinita. O tema em comum é o cálculo de um número que é limite de outro calculado mais facilmente. Este conceito de limite diferencia o cálculo das outras áreas da matemática. Na verdade, podemos definir cálculo como a área da matemática que lida com limites. Após Sir Isaac Newton ter inventado sua versão do cálculo, ele o usou para explicar o movimento dos planetas em torno do sol. Hoje o cálculo é usado para calcular as órbitas dos satélites e espaçonaves, predizer tamanhos de populações, estimar taxa de variação do preço do petróleo, prever mudanças climáticas, medir batimento cardíaco, calcular seguros, e muitas outras áreas. As seguintes perguntas nos dizem o poder desta disciplina:
\begin{enumerate}
  \item Como podemos explicar que o ângulo de elevação de um observador até o topo do arco-íris é $42\degree$?
  \item Como podemos explicar o formato das latas nas prateleiras de um supermercado?
  \item Qual é o melhor lugar para se sentar em um cinema?
  \item Como podemos projetar uma montanha russa de forma que seja totalmente suave?
  \item O quão longe do aeroporto o piloto deve começar o pouso?
  \item Como unir curvas para formar letras numa impressora a laser?
  \item Como estimar o número de trabalhadores necessários para construir a Grande Pirâmide de Khufu no Egito Antigo?
  \item O que leva mais tempo: atingir o máximo ao lançar a bola ou ela voltar à altura original?
  \item Como podemos explicar o fato dos planetas e satélites se moverem em órbitas elípticas?
  \item Como distribuir o fluxo de água nas turbinas de uma estação hidroelétrica para maximizar a produção de energia?
  \item Se um bola de gude, bola de squash, uma barra de metal e um cano de chumbo rolarem numa rampa, qual deles chega primeiro no solo?
\end{enumerate}

\section{Funções e limites}
\subsection{Quatro formas de representar uma função}
Funções aparecem quando quantidades dependem de outras. Considere essas 4 situações:
\begin{itemize}
  \item A área $A$ de um círculo depende do raio $r$. A regra que conecta $r$ e $A$ é dada por: $A=\pi r^2$. Associa-se cada valor positivo de $r$ a um valor de $A$. Assim, $A$ é uma \emph{função} de $r$.
  \item A população mundial $P$ depende do tempo $t$. A Tabela abaixo mostra estimativas de $P(t)$ no tempo $t$, para alguns anos. Por exemplo, $P(1950)\approx 2.560.000.000$. Para cada valor de $t$ há um correspondente de $P$ e $P$ é em função de $t$.\begin{table}[!ht]
    \centering
    \vspace{-0.25cm}
    \setlength\tabcolsep{0.1cm}
    \begin{tabular}{|>{\centering\columncolor{bookbluearea}}m{1.5cm}|c|c|c|c|c|c|c|c|}\hline
      Ano&1940&1950&1960&1970&1980&1990&2000&2010\\\hline
      População (milhões)&2300&2560&3040&3710&4450&5280&6080&6870\\\hline
    \end{tabular}
    \vspace{-0.25cm}
  \end{table}
  \item O custo $C$ de enviar um envelope depende do peso $p$. Embora não seja simples conectar $p$ e $C$, os correios definem uma regra para determinar $C$ quando $p$ é conhecido.
  \item O valor $a$ de um acelerômetro de um smartphone é uma função do tempo $t$. Figura~\ref{fig:earthquake} mostra um gráfico da função. Para um dado valor de $t$, o gráfico mostra o valor de $a$.\vspace{-0.2cm}\begin{figure}[!ht]
    \centering
    \includegraphics{calculus/northridge-earthquake}
    \caption{Aceleração do movimento de um smartphone.}
    \label{fig:earthquake}
  \end{figure}
\end{itemize}
Em todos os casos, dizemos que o segundo valor ($A$, $P$, $C$ ou $a$) é uma função do primeiro ($r$,$t$,$p$ ou $t$). Uma \textbf{função} $f$ é uma regra que atribui a cada elemento $x$ em um conjunto $D$ exatamente um elemento, chamado $f(x)$, em um conjunto $E$. De forma compacta, $f:D\rightarrow E$. Geralmente $D=\mathds{R}$ e $E=\mathds{R}$ (números reais). $D$ é o \textbf{domínio} da função e $E$ é o \textbf{contradomínio}. $f(x)$ é o \textbf{valor de $f$ em $x$} e se lê "$f$ de $x$". O conjunto de todos os elementos $y$ do contradomínio os quais existe $x$ no domínio tal que $f(x)=y$ denomina-se \textbf{imagem}. O símbolo que representa um número no domínio é a \textbf{variável independente}. O símbolo que representa um número na imagem é a \textbf{variável independente}.

Uma função também pode ser interpretada como uma \textbf{máquina} (Figura~\ref{fig:machine-diagram}), aceitando um elemento $x$ do domínio da função $f$ na entrada e devolvendo a imagem $f(x)$ na saída. Um exemplo são as funções da calculadora.
\begin{figure}[!ht]
  \centering
  \vspace{-0.3cm}
  \begin{minipage}[b]{0.3\columnwidth}
    \centering
    \subfloat[\label{fig:machine-diagram}]{\includegraphics[width=\columnwidth]{calculus/machine-diagram}}%
    \vfill
    \subfloat[\label{fig:arrow-diagram}]{\includegraphics[width=\columnwidth]{calculus/arrow-diagram}}%
  \end{minipage}
  \sbox{\measurebox}{%
    \begin{minipage}[b][\ht\measurebox]{0.68\columnwidth}
      \subfloat[\label{fig:graph}]{\includegraphics[width=0.54\columnwidth]{calculus/graph}}\hspace{0.01cm}
      \subfloat[\label{fig:graph2}]{\includegraphics[width=0.41\columnwidth]{calculus/graph2}}
  \end{minipage}}
  \usebox{\measurebox}\qquad
  \caption{Função como: (a) máquina, (b) diagrama de setas e (c) gráfico. (d) Domínio e imagem de uma função.}
\end{figure}

Outra interpretação de $f$ é como um \textbf{diagrama de setas} (Figura~\ref{fig:arrow-diagram}), conectando domínio $D$ e imagem $E$, associando $a$ e $f(a)$, $x$ e $f(x)$, etc. Por fim, a interpretação mais comum é como um \textbf{gráfico} (Figura~\ref{fig:graph}), mostrando pares ordenados $\left\{(x,f(x)):x\in D\right\}$ como pontos (x,y) num plano de coordenadas, ilustrando a "história de vida" de $f$ e podendo indicar também o domínio e a imagem (Figura~\ref{fig:graph2}).

\example{1} Seja $f$ do gráfico da Figura~\ref{fig:graph}.
\begin{enumerate}[label=(\alph*)]
  \item Encontre os valores de $f(1)$ e $f(5)$.
  \item Qual é o domínio e imagem de $f$?
\end{enumerate}
\solution
\begin{enumerate}[label=(\alph*)]
  \item Vemos que o ponto (1, 3) está no gráfico de $f$, então $f(1) = 3$. Contando 5 unidades no eixo $x$, $f(5)$ se situa cerca de 0.7 unidades abaixo do eixo $y$ ($f(5)\approx-0.7$).
  \item $f(x)$ é definido quando $0\leq x \leq 7$, então o domínio é o intervalo [0,7]. Veja que f é definido quando $-2\leq y \leq 4$, então a imagem é o intervalo [-2,4].
\end{enumerate}
\vspace{-0.2cm}
\exampleEnd
\example{2} Desenhe o gráfico e ache o domínio e imagem de:
\vspace{-0.4cm}\begin{multicols}{2}
  \begin{enumerate}[label=(\alph*)]
    \item $f(x)=2x-1$
    \item $g(x)=x^2$
  \end{enumerate}
\end{multicols}\vspace{-0.3cm}
\solution
\begin{enumerate}[label=(\alph*)]
  \item A equação do gráfico é $y=2x-1$, uma linha de declive 2 e intercepto -1. Figura~\ref{fig:ex2-line} mostra o gráfico de $f$. A expressão $2x-1$ é definida para todos os números reais, então o domínio de $f$ é $\mathds{R}$. Veja que a imagem também é $\mathds{R}$.
  \item Como $g(2)= 2^2 = 4$ e $g(-1) = (-1)^2 = 1$, podemos plotar os pontos $(2,4)$ e $(-1,1)$, e uni-los com outros pontos no gráfico (Figura~\ref{fig:ex2-parabola}). A equação do gráfico é $y=x^2$, que representa uma parábola. O domínio de $g$ é $\mathds{R}$. A imagem de $g$ consiste em todos os valores de $g(x)$, ou seja, todos os números na forma $x^2$. Mas $x^2\geq 0, \forall x$. Assim, a imagem de $g$ é $\{y:y\geq 0\} = [0,\infty)$.\vspace{-0.3cm}
\end{enumerate}
\begin{figure}[!ht]
  \centering
  \subfloat[Linha\label{fig:ex2-line}]{\includegraphics{calculus/ex2-line}}%
  \subfloat[Parábola\label{fig:ex2-parabola}]{\includegraphics{calculus/ex2-parabola}}%
  \caption{Gráficos do Exemplo 2}
\end{figure}
\vspace{-0.6cm}
\exampleEnd
\example{3} Se $f(x)=2x^2-5x+1$ e $h\neq 0$, avalie $\displaystyle\frac{f(a+h)-f(a)}{h}$, o qual é denominada \textbf{quociente diferencial}, a taxa de variação de f(x) entre $x=a$ e $x=a+h$

\solution Obtemos $f(a+h)$ trocando $x$ por $a+h$ na equação de $f(x)$:
$$
\begin{aligned}
f(a+h) &=2(a+h)^2-5(a+h)+1\\
&= 2(a^2+2ah+h^2)-5(a+h)+1\\
&=2a^2+4ah+2h^2-5a-5h+1
\end{aligned}
$$
Então substituímos na expressão dada e o simplificamos: $$
\begin{aligned}
\frac{f(a+h)-f(a)}{h} &= \frac{(2a^2+4ah+2h^2-5a-5h+1)-(2a^2-5a+1)}{h}\\
&=\frac{2a^2+4ah+2h^2-5a-5h+1-2a^2+5a-1}{h}\\
&=\frac{4ah+2h^2-5h}{h}=4a+2h-5
\end{aligned}
$$

\vspace{-0.1cm}
\exampleEnd
\vspace{-0.9cm}

\subsubsection{Representação de funções}
Há 4 formas de representar uma função: verbalmente (por uma descrição em palavras), numericamente (por uma tabela de valores), visualmente (por um gráfico) e algebricamente (por uma fórmula explícita). É interessante saber como mudar representações de uma dada função (No Exemplo 2 iniciamos com fórmulas algébricas e obtivemos os gráficos).

A representação mais usada da área do círculo como função do raio é provavelmente a fórmula algébrica $A(r)=\pi r^2$, embora seja possível compilar uma tabela de valores ou desenhar um gráfico (metade uma parábola). O domínio é $\{r:r>0\}=(0,\infty)$ e a imagem também é $(0,\infty)$.

Um exemplo de descrição verbal: $P(t)$ é a população humana mundial no tempo $t$. Seja $t=0$ correspondendo ao ano 1900. Podemos gerar uma tabela de valores e em seguida um gráfico de dispersão (Figura~\ref{fig:population-scatter}), que nos ajuda a entender todos os dados de uma vez. Uma fórmula que descreva de forma exata a população pode ser impossível. Mas podemos encontrar uma fórmula que se aproxima de $P(t)$: $P(t) \approx f(t) = (1,43653\times 10^9)\cdot(1,01395)^t$.
\vspace{-0.5cm}\begin{table}[!ht]
  \setlength\tabcolsep{0.04cm}
  \begin{tabular}{|cccccccccccc|}\hline
    \multicolumn{12}{|c|}{$t$ (anos desde 1900)}\\
    0&10&20&30&40&50&60&70&80&90&100&110\\
    1650&1750&1860&2070&2300&2560&3040&3710&4450&5280&6080&6870\\
    \multicolumn{12}{|c|}{População (milhões)}\\\hline
  \end{tabular}\vspace{-1cm}
\end{table}%
\begin{figure}[!ht]
  \centering
  \subfloat[\label{fig:population-scatter}]{\includegraphics{calculus/population-scatter}}%
  \subfloat[\label{fig:population-scatter-fit}]{\includegraphics{calculus/population-scatter-fit}}%
  \caption{Gráfico (a) de dispersão da população mundial e (b) da curva aproximada}\vspace{-0.3cm}
\end{figure}

Figura~\ref{fig:population-scatter-fit} mostra que $f(t)$ é um bom "ajuste". A função $f$ é um \emph{modelo matemático} para o crescimento populacional. Em outras palavras, é uma função com uma fórmula explícita que aproxima o comportamento de uma dada função. Contudo, veremos que as ideias do cálculo podem ser aplicadas a uma tabela de valores; uma fórmula explícita não é necessária.


A função $P$ é típica de funções que aparecem no mundo real. Começamos com uma descrição verbal (uma hipótese), depois realizamos observações e medições utilizando aparatos e métodos científicos, criando uma tabela de valores. Em seguida podemos derivar uma função aproximada ou aplicar operações de cálculo, mesmo que não tenhamos o total conhecimento do comportamento da função.

Outro exemplo descrito em palavras: seja $C(p)$ o custo de enviar um envelope de peso $p$. A regra dos Correios para Carta Comercial (preço básico) é: 
\vspace{-0.3cm}\begin{table}[!ht]
  \setlength\tabcolsep{0.04cm}
  \centering
	\begin{tabular}{|c|c|c|c|c|}\hline
    \multicolumn{5}{|c|}{Peso $p$ (gramas)}\\\hline
		$p \leq 20$ & $20 > p \leq 50$ & $50 > p \leq 100$ & $100 > p \leq 150$ & $150 > p \leq 200$\\\hline
		1,80&2,55&3,50&4,25&5,05\\\hline
    \multicolumn{5}{|c|}{Custo $C(p)$ (R\$)}\\\hline
	\end{tabular}
  \vspace{-0.5cm}
\end{table}

\example{4} Quando você abre uma torneira de água quente, a temperatura $T$ da água depende de quanto tempo ela esteve correndo. Desenhe um esboço do gráfico de $T$ como uma função do tempo $t$ corrido desde que a torneira foi aberta.

\solution A temperatura inicial da água é perto da temperatura da sala por permanecer no cano. Quando a água do tanque de água-quente começa a fluir, $T$ aumenta rapidamente. Na próxima fase, $T$ se torna constante. Quando o tanque é drenado, $T$ é reduzido à temperatura do abastecimento de água (Figura~\ref{fig:hot-water-faucet}).

\begin{figure}[!ht]
  \centering
  \subfloat[\label{fig:hot-water-faucet}]{\includegraphics{calculus/hot-water-faucet}}%
  \subfloat[\label{fig:storage-container}]{\includegraphics{calculus/storage-container}}%
  \caption{(a) Torneira de água quente e (b) Container de armazenamento.}
  \vspace{-0.5cm}
\end{figure}
\vspace{-0.3cm}
\exampleEnd
\example{5} Um container de armazenamento retangular com o topo aberto tem um volume de 10$\si\meter^3$. O comprimento de sua base é o dobro da sua largura. O material para a base custa R\$10 por $\si{\meter}^2$; o material para os lados custa R\$6 por $\si{\meter}^2$. Expresse o custo dos materiais como uma função da largura da base.

\solution Desenhemos um diagrama como na Figura~\ref{fig:storage-container} e seja $l$ e $2l$ respectivamente a largura e comprimento da base, e $a$ sua altura. A área da base é $(2l)l=2l^2$, então o custo, em reais, do material para a base é $20l^2$. Dois lados tem área $la$ e os outros dois tem área $2la$. Então os 4 lados custam $6(2la + 4la)=36la$. O custo total é $C=20l^2+36la$. Como a caixa tem $10\si\meter^3$ de volume, então $l(2l)a = 10$. Assim, $\displaystyle a=\frac{10}{2l^2}=\frac{5}{l^2}$. Substituindo em $C$: $$C = 20l^2+36l\left(\frac{5}{l^2}\right) = 20l^2+\frac{180}{l}$$.
Assim a equação seguinte expressa $C$ em função apenas de $l$:
$$C(l)=20l^2+\frac{180}{l}\hspace{0.5cm} l > 0$$
\vspace{-0.6cm}

\exampleEnd
\example{6} Encontre o domínio de cada função
\vspace{-0.3cm}\begin{multicols}{2}
  \begin{enumerate}[label=(\alph*)]
    \item $\displaystyle f(x) = \sqrt{x+2}$
    \item $\displaystyle g(x)=\frac{1}{x^2-x}$
  \end{enumerate}
\end{multicols}
\vspace{-0.3cm}\solution 
\begin{enumerate}[label=(\alph*)]
  \item Como a raiz quadrada de um número negativo não é definido (como um número real), o domínio de $f$ consiste em todos os valores de $x$ tais que $x+2 \geq 0$, ou seja, $x \geq -2$. Então o domínio é o intervalo $[-2,\infty)$
  \item Como $\displaystyle g(x)=\frac{1}{x^2-x}=\frac{1}{x(x-1)}$, e divisão por 0 não é permitida, $g(x)$ não é definida quando $x=0$ ou $x=1$. Então o domínio de $g$ é $\{x:x\neq 0, x\neq 1\}=(-\infty,0)\cup(0,1)\cup(1,\infty)$.
\end{enumerate}
\vspace{-0.2cm}
\exampleEnd
O gráfico de uma função no plano-$xy$ é uma curva. Mas todas as curvas no plano-$xy$ são gráficos de funções?
\vspace{-0.3cm}
\subsubsection{Teste da linha vertical}

A curva no plano $xy$ é o gráfico de uma função se e somente se nenhuma linha vertical intersecta a curva mais de uma vez. Caso contrário, a função mapeia um elemento do domínio para mais de um elemento da imagem, o que não pode acontecer. Por exemplo, a parábola $x = y^2-2$ da Figura~\ref{fig:vertical-test-parabola} não é o gráfico de uma função pois há linhas verticais intersectando a parábola duas vezes. Na verdade, tal parábola é um gráfico de duas funções. Veja que $y^2=x+2$ e $y=\pm\sqrt{x+2}$, então a metade superior representa $f(x) = \sqrt{x+2}$ (Figura~\ref{fig:vertical-test-parabola-half}); a inferior, $g(x)=-\sqrt{x+2}$ (Figura~\ref{fig:vertical-test-parabola-other-half}). Revertendo os papeis de x e y (x sendo uma função h(y)), a parábola original se torna o gráfico da função h(y).
\vspace{-0.6cm}\begin{figure}[!ht]
	\centering
	\subfloat[$x=y^2-2$\label{fig:vertical-test-parabola}]{\includegraphics{calculus/vertical-test-parabola}}%
	\subfloat[$y=\sqrt{x+2}$\label{fig:vertical-test-parabola-half}]{\includegraphics{calculus/vertical-test-parabola-half}}%
	\subfloat[$y=-\sqrt{x+2}$\label{fig:vertical-test-parabola-other-half}]{\includegraphics{calculus/vertical-test-parabola-other-half}}%
	\caption{(a) Não é um gráfico de uma função, mas (b) e (c) são.}
  \vspace{-0.5cm}
\end{figure}

\subsubsection{Funções definidas em trechos}

Uma função não pode ter duas saídas para a mesma entrada (pelo teste da linha vertical), mas ela pode ter mais de uma fórmula aplicadas a diferentes partes do domínio.

\example{7} Seja a função $\displaystyle f(x) = \begin{cases}
1-x, & \text{ se } x \leq -1\\
x^2, & \text{ se } x > -1\\
\end{cases}$.
Avalie $f(-2)$, $f(-1)$ e $f(0)$ e desenhe o gráfico.

\solution
Este é um exemplo de função com duas regras aplicadas em faixas diferentes no domínio. Se $x \leq 1$, a função aplica a primeira regra, aplicando a segunda caso contrário ($x > 1$). Como $-2 \leq -1$, então $f(-2) = 1-(-2) = 3$. Como $-1\leq -1$, então $f(-1)=1-(-1)=2$. Como $0 > -1$, então $f(0) = 0^2 = 0$. Para desenhar o gráfico completo, desenhe separadamente cada regra, como na Figura~\ref{fig:piecewise-parabola}. O \textbf{ponto fechado} (preenchido) indica o que a coordenada $(-1,2)$ está incluída no gráfico, ou seja, que $f(-1) = 2$. O \textbf{ponto aberto} (vazio) indica que $(-1,-1)$ não está incluída no gráfico. Só precisamos desenhar estes pontos se houver uma descontinuidade no gráfico.
\vspace{-0.7cm}\begin{figure}[!ht]
  \centering
  \subfloat[\label{fig:piecewise-parabola}]{\includegraphics[width=0.276\columnwidth]{calculus/piecewise-parabola}}
  \subfloat[\label{fig:absolute-value}]{\includegraphics[width=0.29\columnwidth]{calculus/absolute-value}}%
  \subfloat[\label{fig:ex9}]{\includegraphics[width=0.42\columnwidth]{calculus/ex9}}%
  \caption{Gráfico: (a) Exemplo 7, (b) Exemplo 8 e (c) Exemplo 9.}
  \vspace{-0.7cm}
\end{figure}
\exampleEnd
\example{8} Desenhe a função de valor absoluto $f(x) = |x|$.

\solution
Embora pareça apenas uma fórmula, temos duas. A imagem é $|x|\geq 0, \forall x$. Devemos transformar os números negativos do domínio em positivos. Para os já positivos, não precisamos fazer nada. Por isso: $$\displaystyle |x|=\begin{cases}
x &\text{ se } x \geq 0\\
-x & \text{ se } x < 0
\end{cases}$$

\noindent Podemos desenhar separadamente f(x) = x e f(x) = -x (Figura~\ref{fig:absolute-value}). Não é necessário desenhar os pontos aberto e fechado em $x=0$.
\vspace{-0.7cm}

\exampleEnd
\example{9} Encontre uma fórmula $f$ desenhada na Figura~\ref{fig:ex9}
\solution A linha entre (0,0) e (1,1) tem declive $m = 1$ e y-intercepto $b=0$. Então sua equação é $y=x$. Então a primeira parte da fórmula é $f(x) = x$ se $0\leq x \leq 1$. A linha entre (1,1) e (2,0) tem declive $m=-1$, e como $m=\Delta y/\Delta x=(y-y_0)/(x-x_0)$, onde $(x_0,y_0)$ é um ponto conhecido: $-1 = (y-0)/(x-2) \Rightarrow y=2-x$. Encontramos a segunda regra: $f(x) = 2-x$, se $1<x\leq 2$. Veja que a terceira parte do gráfico coincide com o eixo $x$ para $x>2$, ou seja, $f(x)=0$ se $x>2$. Juntando as regras: $$f(x)=\begin{cases}
x & \text{ se } 0 \leq x \leq 1\\
2-x & \text{ se } 1 < x \leq 2\\
0 & \text{ se } x > 2
\end{cases}$$

%\example{10} No Exemplo C desta seção consideramos o custo $C(p)$ de enviar um envelope de peso $p$. Usando a tabela dos Correios, temos:
\vspace{-0.2cm}
\exampleEnd
\vspace{-1cm}
\subsubsection{Simetria}
\vspace{-0.2cm}
Se $f(-x)=f(x)$ para todo número $x$ no seu domínio, então $f$ é uma \textbf{função par}. Por exemplo, a função $f(x)=x^2$ é par pois $f(-x)=(-x)^2 = x^2 = f(x)$. A interpretação geométrica é que o eixo $y$ se comporta como um espelho, refletindo o gráfico da metade do domínio na outra (Figura~\ref{fig:even-function}).

Se $f(-x)=-f(x)$ para todo $x$ no seu domínio, então $f$ é uma \textbf{função ímpar}. Por exemplo, $f(x) = x^3$ é ímpar pois $f(-x) = (-x)^3 = -x^3 = -f(x)$. O gráfico é simétrico na origem. Teremos o mesmo gráfico se rotacioná-lo a $180\degree$ da origem (Figura~\ref{fig:odd-function}).
\begin{figure}[!ht]
  \centering
  \subfloat[Função par\label{fig:even-function}]{\includegraphics[width=0.4\columnwidth]{calculus/even-function}}
  \subfloat[Função ímpar\label{fig:odd-function}]{\includegraphics[width=0.4\columnwidth]{calculus/odd-function}}
  \caption{}
  \vspace{-0.5cm}
\end{figure}

\example{10} Determine se cada uma das seguintes funções é par, ímpar ou nem par nem ímpar. 

\vspace{-0.35cm}
\begin{multicols}{3}
\vspace{-0.5cm}
  \begin{enumerate}[label=(\alph*)]
    \item $f(x) = x^5 + x$
    \item $g(x) = 1-x^4$
    \item $h(x) = 2x-x^2$
  \end{enumerate}

\end{multicols}  
\vspace{1cm}

\solution Figura~\ref{fig:ex10} ilustra as seguintes respostas:
\begin{enumerate}[label=(\alph*)]
  \item $f(-x)=(-x)^5 + (-x)= -x^5-x=-(x^5+x)=-f(x)$ (ímpar)
  \item $g(-x)=1-(-x)^4=1-x^4=g(x)$ (par)
  \item $h(-x)=2(-x)-(-x)^2=-2x-x^2\neq h(x)$ e $h(-x)\neq h(x)$ (nem par nem ímpar)
\end{enumerate}\vspace{-0.7cm}
\begin{figure}[!ht]
  \centering
  \subfloat[\label{fig:ex10-a}]{\includegraphics[width=0.3\columnwidth]{calculus/ex10-a}}
  \subfloat[\label{fig:ex10-b}]{\includegraphics[width=0.3\columnwidth]{calculus/ex10-b}}
  \subfloat[\label{fig:ex10-c}]{\includegraphics[width=0.3\columnwidth]{calculus/ex10-c}}
  \caption{Gráficos do Exemplo 10.}
  \label{fig:ex10}
  \vspace{-0.7cm}
\end{figure}
\exampleEnd
  \vspace{-0.9cm}
\subsubsection{Funções crescentes e decrescentes}

O gráfico da Figura~\ref{fig:increasing-decreasing} cresce de $A$ a $B$, decresce de $B$ a $C$, e cresce novamente de $C$ a $D$. Diz-se que a função $f$ é crescente no intervalo [a,b], decrescente em [b,c] e crescente de novo em [c,d]. De maneira geral, a função $f$ é \textbf{crescente} ou \textbf{decrescente} num intervalo $I$ se respectivamente $f(x_1) < f(x_2)$ ou $f(x_1) > f(x_2)$, para todo $x_1 < x_2$ em $I$. Veja na Figura~\ref{fig:increasing-decreasing-parabola} que a função $f(x) = x^2$ é decrescente no intervalo $(\infty, 0]$ e crescente no intervalo $[0,\infty)$.
\vspace{-0.5cm}\begin{figure}[!ht]
  \centering
  \subfloat[\label{fig:increasing-decreasing}]{\includegraphics[width=0.515\columnwidth]{calculus/increasing-decreasing}}
  \subfloat[\label{fig:increasing-decreasing-parabola}]{\includegraphics[width=0.33\columnwidth]{calculus/increasing-decreasing-parabola}}
  \caption{(a) $x_1$ e $x_2$ num intervalo crescente, (b) parábola.}
  \vspace{-0.5cm}
\end{figure}

\subsubsection{Exercícios}

\begin{enumerate}[label=\textbf{\arabic*.},leftmargin=*]
  \item Se $f(x)=x+\sqrt{2-x}$ e $g(u)=u+\sqrt{2-u}$, é verdade que $f=g$?
  \item Se $\displaystyle f(x)=\frac{x^2-x}{x-1}$ e $\displaystyle g(x)=x$, é verdade que $f=g$?\vspace{0.6mm}
  \item O gráfico da função $f$ é dada na Figura~\ref{fig:2-1-exercicio3}.
  \begin{enumerate}
    \item Dê o valor de $f(1)$.
    \item Estime o valor de $f(-1)$.
    \item $f(x)=1$ para quais valores de $x$?
    \item Estime o valor de $x$ tal que $f(x)=0$.
    \item Expresse o domínio e imagem de $f$.
    \item Em qual intervalo $f$ é crescente?
  \end{enumerate}\vspace{-0.3cm}
  \begin{figure}[!ht]
    \centering
    \begin{minipage}{0.426\columnwidth}
        \includegraphics[width=\columnwidth]{calculus/ex2-1-6-3}
        \caption{Exercício 3}
        \label{fig:2-1-exercicio3}
    \end{minipage}
    \begin{minipage}{0.40\columnwidth}
        \includegraphics[width=\columnwidth]{calculus/ex2-1-6-4}
        \caption{Exercício 4}
        \label{fig:2-1-exercicio4}
    \end{minipage}
  \end{figure}
  \item[\difficultQuestion] Os gráficos de $f$ e $g$ são dados na Figura~\ref{fig:2-1-exercicio4}.
  \begin{enumerate}
    \item Obtenha os valores de $f(-4)$ e $g(3)$.
    \item $f(x)=g(x)$ para quais valores de $x$?
    \item Estime a solução da equação $f(x)=-1$.
    \item $f$ está decrescendo em qual intervalo?
    \item Obtenha o domínio e imagem de $f$.
    \item Obtenha o domínio e imagem de $g$.
  \end{enumerate}
  \item Nesta seção discutimos exemplos de funções usadas no dia-a-dia: População como função do tempo, custo de postagem como função de peso, temperatura da água como função do tempo. Dê três outros exemplos de funções descritas verbalmente. O que dizer sobre o domínio e imagem de cada uma? Se possível, desenhe um esboço do gráfico de cada uma.
  
  \vspace{0.2cm}
  \rangeItem{6-9}{Determine se a curva é um gráfico de uma função de $x$. Se sim, diga o domínio e imagem dela.}
  \vspace{0.0cm}
  
  \begin{multicols}{2}
  \item \adjustbox{valign=t}{\includegraphics[width=0.9\columnwidth]{calculus/ex2-1-6-7}}
  \item \adjustbox{valign=t}{\includegraphics[width=0.9\columnwidth]{calculus/ex2-1-6-8}}
  \item \adjustbox{valign=t}{\includegraphics[width=0.9\columnwidth]{calculus/ex2-1-6-9}}
  \item \adjustbox{valign=t}{\includegraphics[width=0.9\columnwidth]{calculus/ex2-1-6-10}}
  \end{multicols}
  \item A Figura~\ref{fig:2-1-exercicio10} é um gráfico da temperatura global $T$ durante o século XX. Estime:
  \begin{enumerate}
    \item A temperatura média global em 1950
    \item O ano quando a temperatura média foi $14,2\si\degreeCelsius$
    \item O ano quando a temperatura foi a menor (e a maior).
    \item A imagem de $T$
  \end{enumerate}
    \begin{figure}[!ht]
    \centering
    \begin{minipage}{0.49\columnwidth}
      \includegraphics[width=\columnwidth]{calculus/graph}
      \caption{Exercício 10}
      \label{fig:2-1-exercicio10}
    \end{minipage}
    \begin{minipage}{0.49\columnwidth}
      \includegraphics[width=\columnwidth]{calculus/graph}
      \caption{Exercício 11}
      \label{fig:2-1-exercicio11}
    \end{minipage}
  \end{figure}
  \item Árvores crescem mais rápido e formam anéis mais largos em anos mais quentes e crescem mais devagar e formam anéis mais curtos em anos mais frios. A Figura~\ref{fig:2-1-exercicio11} mostra larguras de aneis de um pinhal Siberiano de 1500 a 2000.
  \begin{enumerate}
    \item Qual a imagem da função de largura do anel?
    \item O que o gráfico nos diz sobre a temperatura da terra? O gráfico reflete as erupções vulcânicas do século XIX?
  \end{enumerate}
  \item[\difficultQuestion] Você poe alguns cubos no copo, enche-o de água fria, e o deixa sobre a mesa. Descreva como a temperatura da água muda conforme o tempo passa. Desenhe o esboço do gráfico da temperatura da água em função do tempo decorrido.
  \item Três corredores competem em uma corrida de $100\si{\meter}$. O gráfico da Figura~\ref{fig:2-1-exercicio12} mostra a distância percorrida por cada corredor em função do tempo. Descreva o que o gráfico diz sobre esta corrida. Quem a venceu? Todos a terminaram?
  \begin{figure}[!ht]
    \centering
    \begin{minipage}{0.49\columnwidth}
      \includegraphics[width=\columnwidth]{calculus/graph}
      \caption{Exercício 13}
      \label{fig:2-1-exercicio12}
    \end{minipage}
    \begin{minipage}{0.49\columnwidth}
      \includegraphics[width=\columnwidth]{calculus/graph}
      \caption{Exercício 14}
      \label{fig:2-1-exercicio13}
    \end{minipage}
  \end{figure}
  \item O gráfico da Figura~\ref{fig:2-1-exercicio13} mostra o consumo de potência por um dia em Setembro em São Francisco-EUA (P é medido em megawatts; t é medido em horas iniciando à meia noite).
  \begin{enumerate}
    \item Qual é o consumo de potência às 6 AM? E às 6 PM?
    \item Quando o consumo foi o menor? E o maior? Há sentido nestes tempos?
  \end{enumerate}
  \item Um avião decola do aeroporto e pousa uma hora depois em outro, 400$\si\km$ distante. Se $t$ é o tempo em minutos desde que o avião deixa o terminal, seja $x(t)$ a distância horizontal percorrida e $y(t)$ a altitude do avião.
  \begin{enumerate}
    \item Desenhe um possível gráfico de $x(t)$.
    \item Desenhe um possível gráfico de $y(t)$.
    \item Desenhe um possível gráfico da velocidade no solo.
    \item Desenhe um possível gráfico da velocidade vertical.
  \end{enumerate}
  \item Leituras da temperatura $T$ (em $\si\degreeCelsius$) foram gravadas a cada duas horas de 0AM às 2PM em São Paulo em 4 de Junho de 2013. O tempo $t$ foi medido em horas desde a meia noite.\begin{table}[!ht]
    \centering
    \vspace{-0.3cm}
    \begin{tabular}{|>{\centering\columncolor{bookbluearea}}m{0.5cm}|c|c|c|c|c|c|c|c|}\hline
      $t$&0&2&4&6&8&10&12&14\\\hline
      $T$&74&69&68&66&70&78&82&86\\\hline
    \end{tabular}
    \vspace{-0.3cm}
  \end{table}
  \begin{enumerate}
    \item Pelos dados, desenhe o gráfico de $T$ como função de $t$.
    \item Use seu gráfico para estimar a temperatura às 9AM.
  \end{enumerate}
  \item Pesquisadores mediram a concentração de álcool no sangue (BAC em $\si{\gram}/\si{\deci\liter}$) de oito adultos do gênero masculino após rápido consumo de $30\si\milli\si\liter$ de etanol (dois drinques)
  \begin{table}[!ht]
    \centering
    \setlength\tabcolsep{0.15cm}
    \vspace{-0.3cm}
    \begin{tabular}{|>{\centering\columncolor{bookbluearea}}m{0.7cm}|c|c|c|c|c|c|c|}\hline
      $t$ ($\si\hour$)&0&0.2&0.5&0.75&1.0&1.25&1.5\\\hline
      BAC&0&0.025&0.041&0.040&0.033&0.029&0.024\\\hline\hline
      $t$ ($\si\hour$)&1.75&2.0&2.25&2.5&3.0&3.5&4.0\\\hline
      BAC&0.022&0.018&0.015&0.012&0.007&0.003&0.001\\\hline
    \end{tabular}
  \vspace{-0.3cm}
  \end{table}
  \begin{enumerate}
    \item Dos dados, desenhe o gráfico do BAC como função de $t$.
    \item Use seu gráfico para descrever como o efeito do álcool varia com o tempo.
  \end{enumerate}
  \item Se $f(x)=3x^2-x+2$, encontre $f(2)$, $f(a)$, $f(-a)$, $f(a+1)$, $2f(a)$, $f(2a)$, $f(a^2)$, $[f(a)]^2$ e $f(a+h)$.
  \item Um balão esférico com raio $r$ infla com volume $V(r)=\frac{4}{3}\pi r^3$. Encontre uma função que representa a quantidade de ar requerida para inflar o balão de um raio de $r~\si{\centi\meter}$ para um raio de $r+1~\si{\centi\meter}$.
  \rangeItem{\hspace{-0.2cm}20-23}{Avalia o quociente diferença e simplifica tua resposta}
  
  \item[\difficultQuestion] $\displaystyle f(x)=4+3x+x^2$, $\displaystyle \frac{f(3+h)-f(3)}{h}$
  \item $\displaystyle f(x)=x^3$, $\displaystyle \frac{f(a+h)-f(a)}{h}$
  \item $\displaystyle f(x)=\frac{1}{x}$, $\displaystyle \frac{f(x)-f(a)}{x-a}$
  \item $\displaystyle f(x)=\frac{x+3}{x+1}$, $\displaystyle \frac{f(x)-f(1)}{x-1}$
  \NoIndent{\hspace{-0.2cm}\rule{1.02\columnwidth}{0.5pt}}
  \clearpage\rangeItem{\hspace{-0.2cm}24-30}{Encontre o domínio das funções}
  \begin{tabular}{@{}m{\dimexpr.5\columnwidth-.5\columnsep}m{\dimexpr.5\columnwidth-.5\columnsep}@{}}
    \item $\displaystyle f(x)=\frac{x+4}{x^2-9}$ &
    \item $\displaystyle f(x)=\frac{2x^3-5}{x^2+x-6}$ \\\vspace{0.2cm}
    \item $\displaystyle f(x)=\sqrt[3]{2t-1}$ &
    \item $\displaystyle f(x)=\sqrt{3-t}-\sqrt{2+t}$\\
    \item $\displaystyle h(x)=\frac{1}{\sqrt[4]{x^2-5x}}$ &
    \item $\displaystyle f(u)=\frac{u+1}{1+\frac{1}{u+1}}$\\\vspace{0.2cm}
    \item $\displaystyle F(p)=\sqrt{2-\sqrt{p}}$
  \end{tabular}
  \NoIndent{\hspace{-0.2cm}\rule{1.02\columnwidth}{0.5pt}}
  \item Encontre o domínio e imagem e desenhe $h(x)=\sqrt{4-x^2}$
  \rangeItem{\hspace{-0.2cm}32-33}{Encontre o domínio e desenhe o gráfico}
  \begin{tabular}{@{}m{\dimexpr.5\columnwidth-.5\columnsep}m{\dimexpr.5\columnwidth-.5\columnsep}@{}}
      \vspace{0.2cm}\item $\displaystyle f(x)=1,6x-2,4$ &
      \item $\displaystyle g(x)=\frac{t^2-1}{t+1}$
  \end{tabular}
  \NoIndent{\hspace{-0.2cm}\rule{1.02\columnwidth}{0.5pt}}
  \rangeItem{\hspace{-0.2cm}34-37}{Avalie $f(-3), f(0)$, e $f(2)$ para a função definida em trechos. Então desenhe o gráfico da função.}
  \begin{tabular}{@{}m{\dimexpr.5\columnwidth-.5\columnsep}m{\dimexpr.5\columnwidth-.5\columnsep}@{}}
  \item $\displaystyle f(x)=\begin{cases}
  x+2 & \text{ se } x<0\\
  1-x & \text{ se } x\geq 0
  \end{cases}$&
  \item $\displaystyle f(x)=\begin{cases}
  3-\frac{1}{2}x & \text{ se } x<2\\
  2x-5 & \text{ se } x\geq 2
  \end{cases}$\\
  \item $\displaystyle f(x)=\begin{cases}
  x+1 & \text{ se } x\leq -1\\
  x^2 & \text{ se } x>-1
  \end{cases}$&
  \item $\displaystyle f(x)=\begin{cases}
  -1  & \text{ se } x\leq 1\\
  7-2x & \text{ se } x>1
  \end{cases}$
  \end{tabular}
  \NoIndent{\hspace{-0.2cm}\rule{1.02\columnwidth}{0.5pt}}
  \rangeItem{\hspace{-0.2cm}38-43}{Desenhe o gráfico da função}
  \begin{tabular}{@{}m{\dimexpr.5\columnwidth-.5\columnsep}m{\dimexpr.5\columnwidth-.5\columnsep}@{}}
    \item $\displaystyle f(x)=x+|x|$&
    \item $\displaystyle f(x)=|x+2|$\\
    \item $\displaystyle g(t)=|1-3t|$&
    \item $\displaystyle h(t)=|t|+|t+1|$\\
    \item $\displaystyle f(x)=\begin{cases}
    |x| & \text{ se } |x|\leq 1\\
    1 & \text{ se } |x| > 1
    \end{cases}$&
    \item $\displaystyle g(x)=\left||x|-1\right|$
  \end{tabular}
  \NoIndent{\hspace{-0.2cm}\rule{1.02\columnwidth}{0.5pt}}
  \rangeItem{\hspace{-0.2cm}44-49}{Encontre uma expressão para a função cujo gráfico é a curva dada}\vspace{0.1cm}
  \item O segmento de linha conectando os pontos $(1,-3)$ e $(5,7)$
  \item O segmento de linha conectando os pontos $(-5,10)$ e $(7,10)$
  \item[\difficultQuestion] A metade de baixo da parábola $x+(y-1)^2=0$
  \item A metade de cima do círculo $x^2+(y-2)^2=4$
  
  \begin{tabular}{@{}m{\dimexpr.5\columnwidth-.5\columnsep}m{\dimexpr.5\columnwidth-.5\columnsep}@{}}
    \item \adjustbox{valign=t}{\includegraphics[width=0.42\columnwidth]{calculus/ex2-1-6-7}} &
    \item \adjustbox{valign=t}{\includegraphics[width=0.42\columnwidth]{calculus/ex2-1-6-7}}
  \end{tabular}
  \NoIndent{\hspace{-0.2cm}\rule{1.02\columnwidth}{0.5pt}}
  \rangeItem{\hspace{-0.2cm}50-54}{Encontre uma fórmula e defina o domínio}
  \item Um retângulo tem perímetro $20\si{\meter}$. Expresse a área do retângulo como função do comprimento de um dos lados.
  \item Um retângulo tem área $16\si{\meter}^2$. Expresse o perímetro do retângulo como função do comprimento de um dos lados.
  \item Expresse a área de um triângulo equilátero como função do comprimento de um lado.
  \item O comprimento de uma caixa retangular fechada com volume $24~\si{\centi\meter}^3$ é o dobro da sua largura. Expresse a altura da caixa como função da largura.
  \item[\difficultQuestion] Uma caixa regular aberta com volume $2\si{\meter}^3$ tem uma base quadrada. Expresse a área da superfície da caixa como função do comprimento de um lado da base.
  \vspace{-0.25cm}\NoIndent{\hspace{-0.2cm}\rule{1.02\columnwidth}{0.5pt}}
  \vspace{-0.05cm}
  \item Uma janela de Norman tem o formato retangular encimado por semicírculo (Figura~\ref{fig:2-1-exercicio55}). Se o perímetro da janela é 1cm, expresse a área $A$ da janela como função da largura $x$.
  \item Uma caixa com o topo aberto pode ser construído a partir de um pedaço de cartolina com dimensões $30\si{\centi\meter}\times51\si{\centi\meter}$ removendo quadrados iguais de lado $x$ em cada canto e dobrando os lados (Figura~\ref{fig:2-1-exercicio56}). Expresse o volume $V$ como função de $x$.
  \begin{figure}[!ht]
    \centering
    \begin{minipage}{0.49\columnwidth}
      \includegraphics[width=\columnwidth]{calculus/graph}
      \caption{Exercício 13}
      \label{fig:2-1-exercicio55}
    \end{minipage}
    \begin{minipage}{0.49\columnwidth}
      \includegraphics[width=\columnwidth]{calculus/graph}
      \caption{Exercício 14}
      \label{fig:2-1-exercicio56}
    \end{minipage}\vspace{-1.5cm}
  \end{figure}
  \item Um plano de telefone móvel tem uma cobrança básica de R\$50 ao mês. O plano inclui 400 minutos e cobra 30 centavos por cada minuto de uso adicional. Escreva o custo mensal $C$ como função do número $x$ de minutos usados e desenhe como função de $x$ para $0\leq x\leq 600$.
  \item Em uma região a velocidade máxima permitida nas estradas é $150~\si{\kilo\meter/\hour}$ e a mínima é $60~\si{\kilo\meter/\hour}$. A multa por violar tais limites é $R\$30$ por cada $\si{\kilo\meter/\hour}$ acima do máximo ou abaixo do mínimo. Expresse o valor da multa $M$ como função da velocidade $x$ e desenhe $M(x)$ para $0\leq x\leq 200$.
  \item Uma companhia de eletricidade cobra de seus consumidores uma taxa básica de 10R\$ ao mês, mais 6 centavos por kilowatt-hora (kWh) para os primeiros 1200kWh e 7 centavos por kWh para qualquer uso acima de 1200kWh. Expresse o custo mensal $E$ como função da quantidade $x$ de potência elétrica usada. Então desenhe a função $E$ para $0\leq x\leq 2000$.
  \item[\difficultQuestion] Em um certo país, o imposto salarial é definido como se segue. Não há imposto até R\$10.000. Para qualquer salário acima de R\$10.000 até R\$20.000, a taxa é de 10\%. Para salários acima de R\$20.000, a taxa é de 5\%.
  \begin{enumerate}
    \item Desenhe o gráfico da taxa $T$ como função do salário $s$.
    \item Qual o valor do imposto para R\$14.000? E R\$26.000?
    \item Plote o gráfico do imposto $I$ como função do salário $s$.
  \end{enumerate}
  \rangeItem{\hspace{-0.2cm}61-62}{Os gráficos de $f$ e $g$ são mostrados. Decida se cada função é par, ímpar ou nenhum. Explique seu raciocínio.}
  \begin{tabular}{@{}m{\dimexpr.5\columnwidth-.5\columnsep}m{\dimexpr.5\columnwidth-.5\columnsep}@{}}
    \item \adjustbox{valign=t}{\includegraphics[width=0.42\columnwidth]{calculus/ex2-1-6-7}} &
    \item \adjustbox{valign=t}{\includegraphics[width=0.42\columnwidth]{calculus/ex2-1-6-7}}
  \end{tabular}
  \NoIndent{\hspace{-0.2cm}\rule{1.02\columnwidth}{0.5pt}}
  \item \begin{enumerate}
    \item Se o ponto $(5,3)$ está no gráfico de uma função par, qual outro ponto deve também estar?
    \item Se o ponto $(5,3)$ está no gráfico de uma função ímpar, qual outro ponto deve também estar?
  \end{enumerate}
  \item Uma função $f$ tem domínio [-5,5] e uma porção do seu gráfico é mostrado.
  \begin{enumerate}
    \item Complete o gráfico de $f$ se é sabido ser par.
    \item Complete o gráfico de $f$ se é sabido ser ímpar.
  \end{enumerate}
  \rangeItem{\hspace{-0.2cm}65-70}{Determine se $f$ é par, ímpar ou nenhum. Se você tem uma calculadora ou software gráficos, use-a para checar suas respostas visualmente}
  \begin{tabular}{@{}m{\dimexpr.5\columnwidth-.5\columnsep}m{\dimexpr.5\columnwidth-.5\columnsep}@{}}
    \item $\displaystyle f(x)=\frac{x}{x^2+1}$&
    \item $\displaystyle f(x)=\frac{x^2}{x^4+1}$\\\vspace{1mm}
    \item $\displaystyle f(x)=\frac{x}{x+1}$&
    \item $\displaystyle f(x)=x|x|$\\\vspace{1mm}
    \item $\displaystyle f(x)=1+3x^2-x^4$&
    \item $\displaystyle f(x)=1+3x^3-x^5$\\\vspace{1mm}
  \end{tabular}
  \vspace{-0.8cm}\NoIndent{\hspace{-0.2cm}\rule{1.02\columnwidth}{0.5pt}}
  \item Se $f$ e $g$ são funções pares, $f+g$ é par? Se $f$ e $g$ são ímpares, $f+g$ é ímpar? E se $f$ é par e $g$ é ímpar? Explique.
  \item Se $f$ e $g$ são funções pares, o produto $fg$ é par? Se $f$ e $g$ são ímpares, $fg$ é ímpar? E se $f$ é par e $g$ é ímpar? Explique.
\end{enumerate}

\subsection{Modelos matemáticos e funções essenciais}

Um \textbf{modelo matemático} é uma descrição matemática de um fenômeno do mundo real como o tamanho da população, a demanda de um produto, a velocidade de um objeto em queda, a concentração de um produto em uma reação química, a expectativa de vida de uma pessoa, ou a economia de reduções de emissões. O propósito do modelo é entender o fenômeno e talvez fazer predições sobre o comportamento futuro.

Figura~\ref{fig:modeling-process} mostra o processo de modelagem matemática. Dado um problema do mundo real, nossa primeira tarefa é formular um modelo matemático identificando e nomeando as variáveis dependentes e independentes e fazendo premissas que simplificam o fenômeno o suficiente para deixá-lo matematicamente tratável. Usamos nosso conhecimento sobre o fato e de nossas habilidades matemáticas para obter equações que relacionam as variáveis. Em situações onde não há lei física para nos guiar, precisaremos coletar dados (seja de um dataset disponível ou conduzindo nossos próprios experimentos) e examiná-los na forma de tabela para encontrar padrões. Desta representação numérica de uma função podemos obter uma representação gráfica que pode já sugerir, em alguns casos, fórmulas algébricas.
\vspace{-0.5cm}\begin{figure}[!ht]
  \centering
  \includegraphics[width=\columnwidth]{calculus/modeling-process}
  \caption{Processo de modelagem de um fenômeno}
  \label{fig:modeling-process}
  \vspace{-0.5cm}
\end{figure}

O segundo passo é aplicar matemática para o modelo matemático a fim de derivar conclusões matemáticas. O terceiro passo é transformar as conclusões matemáticas em informações sobre o fenômeno original e oferecer explicações ou fazer previsões. O passo final é testá-las com novos dados reais. Caso as previsões não retratem a realidade, o modelo é refinado (ou um novo modelo é formulado) e o processo é repetido. Vejamos alguns tipos de funções que podem ser usados para modelar relacionamentos observados no mundo real. Discutiremos características e exemplos de aplicações.\vspace{-0.3cm}

\subsubsection{Modelos Lineares}
O gráfico de uma \textbf{função linear} é uma linha com o formato declive-intercepto: $y=f(x)=mx+b$, onde $m$ é o declive e $b$ o $y$-intercepto. Ele cresce a uma taxa constante ($m$) e a tangente em qualquer ponto coincide com a linha. A função linear $f(x)=3x-2$ é mostrada na Figura~\ref{fig:linear-model}. Quando $x$ aumenta em $0,1$, o valor de $f(x)$ aumenta $0,3$. Assim, $f(x)$ aumenta 3 vezes mais rápido que $x$. Por isso o declive do gráfico de y=3x-2, pode ser interpretado como a taxa de variação de $y$ com respeito a $x$.
\vspace{-0.5cm}\begin{figure}[!ht]
  \centering
  \subfloat[$f(x)=3x-2$ (gráfico e tabela)\label{fig:linear-model}]{
    \begin{minipage}{0.32\columnwidth}
      \centering
      \includegraphics[width=\columnwidth]{calculus/ex2-1-6-7}
    \end{minipage}
    \begin{minipage}{0.33\columnwidth}
      \centering
      \begin{tabular}{|c|c|}\hline
        $x$ & $f(x)=3x-2$\\\hline
        1.0 & 1.0 \\
        1.1 & 1.3 \\
        1.2 & 1.6 \\
        1.3 & 1.9 \\
        1.4 & 2.2 \\
        1.5 & 2.5 \\\hline
      \end{tabular}
    \end{minipage}
  }
  \subfloat[Exemplo 1\label{fig:2-2-example-1}]{
    \begin{minipage}{0.32\columnwidth}
      \centering
      \includegraphics[width=\columnwidth]{calculus/ex2-1-6-7}
    \end{minipage}
  }
  \caption{}\vspace{-0.5cm}
\end{figure}

\example{1}
\begin{enumerate}[label=(\alph*)]
  \item Como o ar seco move para cima, ele se expande e esfria. Se a temperatura no solo é $20\si\degreeCelsius$ e na altura de $1\si\km$ é $10\si\degreeCelsius$, expresse a temperatura $T$ (em $\si\degreeCelsius$) como função da altura $h$ (em $\si\km$), assumindo que um modelo linear é apropriado.
  \item Plote o gráfico da função de (a). O que o declive representa?
  \item Qual é a temperatura na altura de $2,5\si\km$?
\end{enumerate}
\solution
\begin{enumerate}[label=(\alph*)]
  \item Assumindo $T$ uma função linear de $h$, então $T=mh+b$. O ponto $(0,20)$ está no gráfico, então $20=m\cdot 0 + b = b$. O ponto $(1,10)$ também está no gráfico, então $10=m\cdot 1 + 20$. Assim, $m=-10$. Dessa forma: $T=-10h+20$.
  \item O gráfico está na Figura~\ref{fig:2-2-example-1}. O declive é $m=-10\si\degreeCelsius/\si\km$, e é a taxa de variação da temperatura com relação à altura.
  \item Quando $h=2,5$, a temperatura é $T=-10(2,5)+20=-5\si\degreeCelsius$.
  
\end{enumerate}
\exampleEnd
Se não houver lei ou princípio físico para nos ajudar a formular um modelo, construímos um \textbf{modelo empírico}, baseado apenas nos dados coletados. Investigamos uma curva que se ajusta ao dado no sentido de capturar a tendência dos pontos.

\example{2} A tabela seguinte lista o nível de dióxido de carbono na atmosfera, medido em partes por milhão no Observatório de Mauna Loa de 1980 a 2012. Use os dados para encontrar um modelo do nível de dióxido de carbono.
\vspace{-0.3cm}\begin{figure}[!ht]
  \centering
  \subfloat[$f(x)=3x-2$ (gráfico e tabela)\label{fig:2-2-example2}]{
    \begin{minipage}{0.44\columnwidth}
      \centering
      \begin{tabular}{|c|c|}\hline
        $x$ & $f(x)=3x-2$\\\hline
        1.0 & 1.0 \\
        1.1 & 1.3 \\
        1.2 & 1.6 \\
        1.3 & 1.9 \\
        1.4 & 2.2 \\
        1.5 & 2.5 \\\hline
      \end{tabular}
    \end{minipage}
    \begin{minipage}{0.52\columnwidth}
      \centering
      \includegraphics[width=0.7\columnwidth]{calculus/ex2-1-6-7}
    \end{minipage}
  }
  \caption{}\vspace{-0.5cm}
\end{figure}

\solution Com os dados, podemos criar um gráfico de dispersão, onde $t$ é o tempo (em anos) e $C$ o nível de CO$_2$ (em ppm).

Perceba que os dados se aproximam de uma linha, então é natural que escolhamos o modelo linear. Porém há muitas possíveis linhas que aproximam estes pontos. Qual deles escolheremos? Uma forma é escolher a linha que conecta o primeiro e último pontos. O declive desta linha é: $$\frac{393,8-338,7}{2012-1980}=\frac{55,1}{32}=1,721875\approx 1,722$$
Nossa equação é (Figura~\ref{fig:2-2-example2-first-last}): $$C-338,7=1,722(t-1980) \text{ ou } C = 1,722t-3070,86$$
Perceba que nosso modelo dá valores mais altos para a maioria dos níveis CO$_2$. A \textbf{regressão linear} é um método estatístico que fornece um modelo melhor. Pela calculadora gráfica, insira os valores no editor de dados e escolha o comando de regressão linear (Maple contém o comando \verb|fit|/\verb|leastsquare|; Mathematica tem o comando \verb|Fit|). O declive e y-intercepto da linha de regressão são $m=1,71262$ e $b=-3054,14$. Assim nosso modelo de mínimos quadrados para o nível CO$_2$ é $$C=1,71262t-3054,14$$
Figura~\ref{fig:2-2-example2-regression} mostra a linha de regressão.
\vspace{-0.5cm}\begin{figure}[!ht]
  \centering
  \begin{minipage}{0.49\columnwidth}
    \includegraphics[width=\columnwidth]{calculus/graph}
    \caption{Exercício 13}
    \label{fig:2-2-example2-first-last}
  \end{minipage}
  \begin{minipage}{0.49\columnwidth}
    \includegraphics[width=\columnwidth]{calculus/graph}
    \caption{Exercício 14}
    \label{fig:2-2-example2-regression}
  \end{minipage}\vspace{-1.5cm}
\end{figure}

\vspace{0.8cm}
\exampleEnd
\subsubsection{Funções polinomiais}
Uma função $P$ é \textbf{polinomial} se $$P(x)=a_nx^n+a_{n-1}x^{n-1}+\dots+a_2x^2+a_1x+a_0$$
onde $n$ é um inteiro não-negativo e os números $a_0,a_1,a_2,\dots,a_n$ são constantes chamadas de \textbf{coeficientes} do polinômio. O \textbf{grau} do polinômio é o maior $n$ onde $a_n\neq 0$. Por exemplo, o polinômio $P(x)=2x^6-x^4+\frac{2}{5}x^3+\sqrt{2}$ tem grau 6. Um polinômio de grau 1 é da forma $P(x)=mx+b$ (a nossa função linear). Um polinômio de grau 2 é da forma $P(x)=ax^2+bx+c$ e é denominada \textbf{função quadrática}. Seu gráfico é sempre uma parábola obtida transformando a parábola canônica $y=ax^2$. A parábola tem abertura para cima se $a>0$ e para baixo se $a<0$. Um polinômio de grau 3 é da forma P(x)=$ax^3+bx^2+cx+d$, com $a\neq 0$, denominada \textbf{função cúbica}. Figura~\ref{fig:polinomiais} mostra o gráfico de uma função cúbica e polinômios de grau 4 e 5. Veja que o de grau 5 tem quatro mudanças de crescimento, o de grau 4 tem três e o de grau 3 tem duas. Veremos por que eles têm esse formato.
\vspace{-0.5cm}\begin{figure}[!ht]
  \centering
  \subfloat[$y=x^3-x+1$\label{fig:01}]{\includegraphics[width=.33\columnwidth]{calculus/ex2-1-6-7}}
  \subfloat[$y=x^4-3x^2+x$\label{fig:02}]{\includegraphics[width=.33\columnwidth]{calculus/ex2-1-6-7}}
  \subfloat[$y=3x^5-25x^3+60x$\label{fig:03}]{\includegraphics[width=.33\columnwidth]{calculus/ex2-1-6-7}}
  \caption{}\label{fig:polinomiais}
\end{figure}\vspace{-0.5cm}

\example{1} Uma bola é solta da cobertura de observação, $450\si\meter$ acima do solo, e sua altura $h$ acima do solo é gravada a um invervalo de 1-segundo na Tabela 2. Encontre um modelo que aproxime aos dados e o use para predizer o tempo em que a bola cai do solo.

\solution Desenhamos um gráfico de dispersão dos dados e vemos que o modelo linear não é adequado. Todavia, parece semelhante a uma parábola, então tentaremos uma função quadrática. Usando uma calculadora ou software com o método dos quadrados mínimos, obtemos o seguinte modelo: $h=449,36+0,96t-4,90t^2$ (Figura~\ref{fig:01}). A bola atinge o solo quando $h=0$, assim: $-4,90t^2+0,96t+449,36 = 0$. Manipulando algebricamente: $$t=\frac{-0,96\pm \sqrt{(0,96)^2-4(-4,90)(449,36)}}{2(-4,90)}$$
A raiz positiva é $t\approx 9,67$, então a bola atinge o chão cerca de 9,7 segundos a partir do topo do prédio.

\vspace{-0.5cm}\begin{figure}[!ht]
	\centering
	\subfloat[$y=x^3-x+1$\label{fig:04}]{\includegraphics[width=.33\columnwidth]{calculus/ex2-1-6-7}}
	\subfloat[$y=x^4-3x^2+x$\label{fig:05}]{\includegraphics[width=.33\columnwidth]{calculus/ex2-1-6-7}}
	\subfloat[$y=3x^5-25x^3+60x$\label{fig:06}]{\includegraphics[width=.33\columnwidth]{calculus/ex2-1-6-7}}
	\caption{}\label{fig:polinomiais-exemplo1}
\end{figure}
\vspace{-0.7cm}
\exampleEnd
\vspace{-0.7cm}
\subsubsection{Funções de potência}
Uma função de formato $f(x)=x^a$, onde $a$ é uma constante, é denominada \textbf{função de potência}. Considere vários casos:

\noindent \textbf{1) $\boldsymbol{a=n}$, onde $\boldsymbol{n}$ é um inteiro positivo -- } os gráficos de $f(x)=x^n$ para $n=1,2,3,4$ e 5 são mostrados na Figura~\ref{fig:01} (polinômios de um termo). Sabemos o formato dos gráficos de $y=x$ (uma linha sobre a origem com declive 1) e $y=x^2$ (uma parábola).
\vspace{-0.5cm}\begin{figure}[!ht]
	\centering
	\subfloat[$y=x^3-x+1$\label{fig:07}]{\includegraphics[width=.33\columnwidth]{calculus/ex2-1-6-7}}
	\subfloat[$y=x^4-3x^2+x$\label{fig:08}]{\includegraphics[width=.33\columnwidth]{calculus/ex2-1-6-7}}
	\subfloat[$y=3x^5-25x^3+60x$\label{fig:09}]{\includegraphics[width=.33\columnwidth]{calculus/ex2-1-6-7}}
	\caption{}\label{fig:power-functions}\vspace{-0.3cm}
\end{figure}
A forma geral do gráfico] de $f(x) = x^n$  depende se $n$ é par ou ímpar. Se $n$ é par, então $f(x)=x^n$ é uma função par e seu gráfico é similar a uma parábola $y=x^2$. Se $n$ é ímpar, $f(x)=x^n$ é ímpar e seu gráfico é similar a $y=x^3$. À medida em que $n$ cresce, o gráfico fica mais achatado perto de 0 e mais íngreme quando $|x|\geq 1$. 

\noindent \textbf{2) $\boldsymbol{a=1/n}$, onde $\boldsymbol{n}$ é um inteiro positivo -- } a função $f(x)=x^{1/n}=\sqrt[n]{x}$ é uma \textbf{função raiz}. Para $n=2$, ele é a função raiz quadrada $f(x)=\sqrt{x}$, cujo domínio é $[0,\infty)$ e cujo gráfico é a metade superior da parábola $x=y^2$. Para valores pares de $n$, o gráfico de $y=\sqrt[n]{x}$ é semelhante a $y=\sqrt{x}$. Para $n=3$, nós temos a função de raiz cúbica $f(x)=\sqrt[3]{x}$, cujo domínio é $\mathds{R}$ é a metade superior da parábola $x=y^2$. O gráfico de $y=\sqrt[n]{x}$ é similar ao de $y=\sqrt[3]{n}$.
\vspace{-0.5cm}\begin{figure}[!ht]
	\centering
	\subfloat[$y=x^3-x+1$\label{fig:10}]{\includegraphics[width=.33\columnwidth]{calculus/ex2-1-6-7}}
	\subfloat[$y=x^4-3x^2+x$\label{fig:11}]{\includegraphics[width=.33\columnwidth]{calculus/ex2-1-6-7}}
	\subfloat[$y=3x^5-25x^3+60x$\label{fig:12}]{\includegraphics[width=.33\columnwidth]{calculus/ex2-1-6-7}}
	\caption{}\label{fig:root-functions}\vspace{-0.2cm}
\end{figure}

\noindent \textbf{3) $\boldsymbol{a=-1}$}
O gráfico da \textbf{função recíproca} $f(x) = x^{-1} = 1/x$ está mostrado na Figura~\ref{fig:01}. Seu gráfico tem equação $y=1/x$, ou $xy=1$, e é uma hipérbole com os eixos das coordenadas como suas assíntotas. Esta função aparece em Física e Química como na Lei de Boyle, que diz que, quando a temperatura é constante, o volume $V$ de um gás é inversamente proporcional à pressão $P$ ($V=C/P$), onde $C$ é constante. Assim o gráfico de $V$ como função de $P$ tem o mesmo formato do lado direito da Figura~\ref{fig:01}. 

Funções potência também são usadas para modelar relacionamentos espécie-área, iluminação como função da distância da fonte de luz e período de revolução do planeta como função da distância do sol.

\vspace{-0.2cm}
\subsubsection{Funções racionais}

Uma \textbf{função racional} $f$ tem formato $\displaystyle f(x)=\frac{P(x)}{Q(x)}$, onde $P$ e $Q$ são polinômios. O domínio são todos os valores $x$ tais que $Q(x)\neq 0$. Um simples exemplo de uma função racional é a função $f(x)=1/x$, cujo domínio é $\{x:x\neq 0\}$; é a função recíproca desenhada na Figura~\ref{fig:2-1-exercicio10}. A função $\displaystyle f(x)=\frac{2x^4-x^2+1}{x^2-4}$ é uma função racional com domínio $\{x:x\neq \pm 2\}$ (Figura~\ref{fig:01}).

\vspace{-0.3cm}
\subsubsection{Funções algébricas}

Uma \textbf{função algébrica} $f$ é construída usando operações algébricas (como adição, subtração, multiplicação, divisão, potência e raiz) começando por polinomiais. Uma função racional é uma função algébrica. Outros exemplos são:$$f(x)=\sqrt{x^2+1}\hspace{1cm} g(x)=\frac{x^4-16x^2}{x+\sqrt{x}}+(x-2)\sqrt[3]{x+1}$$
Um exemplo de função algébrica ocorre na teoria da relatividade. A massa de uma partícula com velocidade $v$ é $$m=f(v)=\frac{m_0}{\sqrt{1-v^2/c^2}}$$
onde $m_0$ é a massa de repouso da partícula e $c=3.0\times 10^5\si\km/\si\second$ é a velocidade da luz no vácuo.

\subsubsection{Funções trigonométricas}

A variável independente de uma \textbf{função trigonométrica} é o ângulo, relacionando-o com os lados de um triângulo. Tais funções (ex: seno, cosseno, tangente, secante, cossecante, cotangente, etc) são importantes para fenômenos periódicos. Neste documento, o ângulo é em radianos ($180\si\degree=\pi \text{ rad}=3,14159\dots$). O domínio do seno e cosseno é $(-\infty, \infty)$ e a imagem e $\left[-1,1\right]$, ou seja, $|\sin x| \leq 1$ e $|\cos x| \leq 1$. Os zeros da função $\sin x$ ocorrem em múltiplos inteiros de $\pi$, ou seja, $\sin x=0$ quando $x=n\pi, n\in \mathds{N}$.

Seno e cosseno são funções periódicas, ou seja, $\sin(x+2n\pi)=\sin x$ e $\cos(x+2n\pi)=\cos x$ para $n\in\mathds{N}$. Isto é interessante para modelar processos repetitivos como ondas e vibrações. 

\example{5} Qual é o domínio da função $\displaystyle f(x)=\frac{1}{1-2\cos x}$? 

\solution Esta função é definida para todos os valores exceto o que faz o denominador 0. Os valores proibidos de $x$ são: $$1-2\cos x = 0\Leftrightarrow \cos x=\frac{1}{2} \Leftrightarrow x = \frac{\pi}{3}+2n\pi \text{ ou } x=\frac{5\pi}{3}+ 2n\pi$$

\exampleEnd

A função tangente é relacionada ao seno e coseno pela equação $\tan x = \sin x/\cos x$ e seu gráfico está mostrado na Figura~\ref{fig:01}. Ele é indefinido quando $\cos x = 0$, ou seja, quando $x=\pm\pi/2,\pm 3\pi/2,\dots$. Sua imagem é $(-\infty, \infty)$. Perceba que a tangente tem período $\pi$: $\tan x+n\pi = \tan x$.

\subsubsection{Funções exponenciais}

A \textbf{função exponencial} é da forma $f(x)=b^x$, onde a base $b$ é uma constante positiva. Os gráficos de $y=2^x$ e $y=(0.5)^x$ estão na Figura~\ref{fig:2-1-exercicio10}. Em ambos os casos o domínio é $(-\infty, \infty)$ e a imagem é $(0,\infty)$. Crescimento populacional ($b > 1$) e decaimento radioativo ($b < 1$) são aplicações.
\vspace{-0.5cm}\begin{figure}[!ht]
	\centering
	\subfloat[$y=x^3-x+1$\label{fig:13}]{\includegraphics[width=.33\columnwidth]{calculus/ex2-1-6-7}}
	\subfloat[$y=x^4-3x^2+x$\label{fig:14}]{\includegraphics[width=.33\columnwidth]{calculus/ex2-1-6-7}}
	\subfloat[$y=3x^5-25x^3+60x$\label{fig:15}]{\includegraphics[width=.33\columnwidth]{calculus/ex2-1-6-7}}
	\caption{}\label{fig:exponential-functions}
\end{figure}\vspace{-0.5cm}
\subsubsection{Funções logarítmicas}

As \textbf{funções logarítmicas} $f(x)=\log_bx$, onde a base $b$ é uma constante positiva, são funções inversas das funções exponenciais. Figura~\ref{fig:01} mostra os gráficos de 4 funções logarítmicas com várias bases. Em cada caso, o domínio é $(0,\infty)$, a imagem é $(-\infty, \infty)$ e a função cresce devagar quando $x > 1$.

\example{6} Classifique as seguintes funções de acordo com os tipos discutidos
\begin{enumerate}[label=(\alph*)]
  \begin{tabular}{@{}m{\dimexpr.5\columnwidth-.5\columnsep}m{\dimexpr.5\columnwidth-.5\columnsep}@{}}
    \item $f(x) = 5^x$&
    \item $g(x) = x^5$\\
    \item $\displaystyle h(x) = \frac{1+x}{1-\sqrt{x}}$&
    \item $u(t) = 1-t+ 5t^4$
  \end{tabular}
\end{enumerate}

\solution
\begin{enumerate}[label=(\alph*)]
	\item $f(x)=5^x$ é uma função exponencial
	\item $g(x)=x^5$ é uma função de potência, também considerado como um polinômio de grau 5.
	\item $h(x) = \frac{1+x}{1-\sqrt{x}}$ é uma função algébrica.
	\item $u(t)= 1-t+5t^2$  é um polinômio de grau 4
\end{enumerate}

\subsubsection{Exercícios}

Classifique cada função como função potência, função raiz, polinomial (diga o grau), função racional, função algébrica, função trigonométrica, função exponencial ou logarítmica.

\begin{enumerate}[label=\textbf{\arabic*.},leftmargin=*]
  \item \begin{enumerate}
    \item $f(x)=\log_2x$
    \item $g(x)=\sqrt[4]{x}$
    \item $h(x)=\frac{2x^3}{1-x^2}$
    \item $u(t) = 1-1,1t + 2,54t^2$
    \item $v(t) = 5^t$
    \item $w(\theta) = \sin\theta \cos^2\theta$
  \end{enumerate}
  \item \begin{enumerate}
    \item $y = \pi^x$
    \item $y = x^\pi$
    \item $y = x^2(2-x^3)$
    \item $y = \tan t - \cos t$
    \item $y = \frac{s}{1+s}$
    \item $y = \frac{\sqrt{x^3-1}}{1+\sqrt[3]{x}}$
  \end{enumerate}
  \item \begin{enumerate}
    \item $y=x^2$
    \item $y=x^5$
  \end{enumerate}
\end{enumerate}


\subsection{Gerando funções a partir de outras}

\subsubsection{Transformações de funções}

\textbf{Translações:} podemos transladar no eixo $y$. Se $c$ é um número positivo, então o gráfico de $y=f(x)+c$ é o gráfico de $y=f(x)$ movido para cima por uma distância de $c$ unidades. Podemos também transladar no eixo $x$. Se $g(x) = f(x-c)$, onde $c>0$, o valor de $g$ em $x$ é o mesmo do valor de $f$ em $x-c$ ($c$ unidades à esquerda de $x$).

\noindent\textbf{Escala e reflexão:} podemos esticar, encolher e até refletir o gráfico. Se $c > 1$, então o gráfico de $y=cf(x)$ é o gráfico de $f(x)$ esticado por um fator de $c$ unidades na direção vertical. O gráfico de $y=-f(x)$ é o gráfico de $y=f(x)$ refletido sobre o eixo $x$ pois cada ponto $(x,y)$ agora se torna $(x,-y)$. $y=f(xc)$ comprime o gráfico por um fator $c$, pois considerando $g(x) = f(xc)$, $g(1)=f(c)$ e $g(2)=f(2c)$ (considerando $c>1$). Figura~\ref{fig:01} mostra a translação, escala e reflexão.

\vspace{-0.5cm}\begin{figure}[!ht]
	\centering
	\subfloat[$y=x^3-x+1$\label{fig:16}]{\includegraphics[width=.33\columnwidth]{calculus/ex2-1-6-7}}
	\subfloat[$y=x^4-3x^2+x$\label{fig:17}]{\includegraphics[width=.33\columnwidth]{calculus/ex2-1-6-7}}
\caption{}\label{fig:transforming-functions}
\end{figure}\vspace{-0.5cm}

\example{1} Dado o gráfico de $y=\sqrt{x}$, use transformações para desenhar $y=\sqrt{x}-2$, $y=\sqrt{x-2}$, $y=\sqrt{x}$, $y=2\sqrt{x}$ e $y=\sqrt{-x}$.
\solution O gráfico de $y=\sqrt{x}$ está na Figura~\ref{fig:01}. Obtemos $y=\sqrt{x}-2$ de $y=\sqrt{x}$ movend 2 unidades para baixo, $y=\sqrt{x-2}$ movendo 2 unidades para a direita, $y=2\sqrt{x}$ esticando verticalmente pelo fator 2 e $y=\sqrt{-x}$ refletindo sobre o eixo $y$.

\vspace{-0.5cm}\begin{figure}[!ht]
	\centering
	\subfloat[$y=x^3-x+1$\label{fig:18}]{\includegraphics[width=.33\columnwidth]{calculus/ex2-1-6-7}}
	\subfloat[$y=x^4-3x^2+x$\label{fig:19}]{\includegraphics[width=.33\columnwidth]{calculus/ex2-1-6-7}}
	\caption{}\label{fig:transforming-functions-example1}
\end{figure}\vspace{-0.5cm}

\exampleEnd

\example{2} Desenhe o gráfico da função $f(x) = x^2 + 6x + 10$
\solution Completando o quadrado perfeito, $y=f(x)=(x+3)^2+1$. Significa que podemos começar com o gráfico $y=x^2$, mover 3 unidades para a esquerda ($y=(x+3)^2$) e por fim mover 1 unidade para cima.

\example{3} Desenhe os gráficos das seguintes funções:
\begin{enumerate}
	\item $y=\sin 2x$
	\item $y=1-\sin x$
\end{enumerate}
\solution 
\begin{enumerate}
	\item Obtemos o gráfico de $y=\sin 2x$ de $y=\sin x$ encolhendo por um fator de 2. O período passa de $2\pi$ para $\pi$.
	\item Para obter o gráfico de $y=1-\sin x$, refletimos $y=\sin x$ e o movemos 1 unidade para cima.
\end{enumerate}

\example{4} Figura~\ref{fig:01} mostra os gráficos do número de horas de luz por dia como função do tempo em várias latitudes. Dado que Filadélfia está localizado aproximadamente $40\si\degree N$ de latitude, encontre uma função que modele o fenômeno em Filadélfia.

\solution Perceba que a função se assemelha a uma senoide movida e esticada. Olhando a curva azul vemos que 

\subsubsection{Combinações de funções}

Duas funções $f$ e $g$ podem ser combinadas para formar novas funções, $f+g$, $f-g$, $fg$ e $f/g$ de uma maneira similar como adicionamos, subtraímos, multiplicamos e dividimos números. As funções soma e diferença são $(f+g)(x) = f(x)+g(x)$ e $(f-g)(x)=f(x)-g(x)$. Se o domínio de $f$ for $A$ e o de $g$ for $B$, então o domínio de $f+g$ é a intersecção $A\cap B$, pois ambos $f(x)$ e $g(x)$ precisam estar definidos. Por exemplo, o domínio de $f(x) = \sqrt{x}$ é $A=[0,\infty)$, e o domínio de $g(x) = \sqrt{2-x}$ é $B=(-\infty, 2]$, então o domínio de $(f+g)(x)=\sqrt{x}+ \sqrt{2-x}$ é $A\cap B=[0,2]$.

Similarmente, as funções produto e quociente são definidas por $(fg)(x)=f(x)g(x)$ e $\displaystyle\left(\frac{f}{g}\right)(x)=\frac{f(x)}{g(x)}$. O domínio de $fg$ é $A\cap B$. Como não permitimos divisão por 0, o domínio de $f/g é \{x\in A\cap B: g(x)\neq 0\}$. Por exemplo, se $f(x)=x^2 e g(x) = x-1$, então o domínio da função racional $(f/g)(x)=x^2/(x-1)$ é $\{x : x\neq 1\}$, ou $(-\infty, 1) \cup (1, \infty)$.

\textbf{Composição} é outra forma de combinar duas funções. Por exemplo, suponha que $y=f(u)=\sqrt{u}$ e $u = g(x) = x^2+1$. Como $y$ é uma função de $u$, que é uma função de $x$, então $y$ é uma função de $x$: $y=f(u)=f(g(x))=f(x^2+1)=\sqrt{x^2+1}$. O símbolo é $(f\circ g)(x)=f(g(x))$. O domínio de $f\circ g$ é o conjunto de todo $x$ no domínio de $g$ tal que $g(x)$ está no domínio de $g$.

\example{6} Se $f(x) = x^2$ e $g(x) = x-3$, encontre as funções compostas $f\circ g$ e $g\circ f$.

\solution $$\begin{aligned}
	f\circ g &= f(g(x)) = f(x-3) = (x-3)^2\\
	g\circ f &= g(f(x)) = g(x^2) = x^2-3
\end{aligned}$$

\exampleEnd

\example{7} Se $f(x)= \sqrt{x}$ e $g(x) = \sqrt{2-x}$, encontre cada uma das seguintes funções e seus domínios
\begin{enumerate}
	\item f\circ g
	\item g\circ f
	\item 
\end{enumerate}


\subsection{O limite de uma função}

\subsubsection{Limites unilaterais}

\subsubsection{Calculando limites usando as leis de limite}

\subsection{Definição precisa do limite}
\subsubsection{Limites infinitos}
\subsection{Continuidade}

%\section{Derivadas}
%\subsection{Derivadas e taxas de variação}
%\subsubsection{Tangentes}
%\subsubsection{Velocidades}
%\subsubsection{Derivadas}
%\subsubsection{Taxas de variação}
%\subsection{Derivadas como funções}
%\subsubsection{Outras notações}
%\subsubsection{Funções não diferenciáveis}
%\subsubsection{Derivadas de ordem mais alta}
%\subsection{Fórmulas de diferenciação}
%\subsubsection{Novas derivadas a partir de outras}
%\subsubsection{Funções potências gerais}
%\subsection{Derivadas de funções trigonométricas}
%\subsection{A regra da cadeia}
%\subsubsection{Prova da regra da cadeia}
%\subsection{Diferenciação implícita}
%\subsection{Taxas de variação nas ciências sociais e naturais}
%\subsubsection{Física}
%\subsubsection{Química}
%\subsubsection{Biologia}
%\subsubsection{Economia}
%\subsubsection{Outras ciências}
%\subsubsection{Uma simples ideia, muitas interpretações}
%\subsection{Taxas relacionadas}
%\subsection{Aproximações lineares e diferenciais}
%\subsubsection{Aplicações em Física}
%\subsubsection{Diferenciais}
%\section{Aplicações da Diferenciação}
%\subsection{Valores máximo e mínimo}
%\subsection{O teorema do valor médio}
%\subsection{Como derivadas afetam a forma de um gráfico}
%\subsubsection{O que $f'$ nos diz sobre $f$}
%\subsubsection{Valores de extremo locais}
%\subsubsection{O que $f''$ nos diz sobre $f$}
%\subsubsection{Limites infinitos no infinito}
%\subsubsection{Definições precisas}
%\subsection{Sumário de desenho de curva}
%\subsection{Problemas de otimização}
%\subsection{Método de Newton}
%\subsection{Antiderivadas}
%
%\section{Integrais}
%\subsection{Áreas e distâncias}
%\subsubsection{O problema da área}
%\subsubsection{O problema da distância}
%\subsection{A integral definida}
%\subsubsection{Avaliando integrais}
%\subsubsection{A Regra do Ponto Médio}
%\subsubsection{Propriedades da integral definida}
%\subsection{Teorema fundamental do cálculo}
%\subsubsection{Diferenciação e Integração como processos inversos}
%\subsection{Integrais indefinidas e o Teorema da Variação Líquida}
%\subsubsection{Integrais indefinidas}
%\subsubsection{Aplicações}
%\subsection{A Regra da Substituição}
%
%
%\section{Funções inversas}
%\section{Técnicas de integração}
%\section{Aplicações de integração}
%\section{Equações diferenciais}
%\section{Equações paramétricas e coordenadas polares}
%\section{Sequências infinitas e séries}
