% !TeX root = main.tex
\clearpage\part{Cálculo}
\section{Introdução}

\subsection{O problema da área}
Há 2500 anos, os gregos achavam áreas através do método da exaustão, dividindo um polígono em triângulos para achar a área $A$ (Figura~\ref{fig:polygon-area-division}). Mas e a área de uma figura curva?  Figura~\ref{fig:circle-polygon-approximation} mostra que aumentando o número de lados do polígono inscrito, sua área se aproxima da área do círculo. Podemos utilizar esta técnica para encontrar a área de qualquer outra curva (Figura~\ref{fig:curve-approximation}). Este é um problema central para o \emph{cálculo integral}.\vspace{-0.6cm}
\begin{figure}[!ht]
  \subfloat[\label{fig:polygon-area-division}]{%
    \includegraphics{calculus/polygon-area-division}  
  }\vspace{-0.3cm}
  \subfloat[\label{fig:circle-polygon-approximation}]{%
    \includegraphics{calculus/circle-polygon-approximation}
  }\\
  \subfloat[\label{fig:curve-approximation}]{%
    \includegraphics{calculus/curve-approximation}
  }
  
  \caption{(a) $A = A_1+A_2+A_3+A_4+A_5$, (b) Encontrando a área do círculo através da aproximação do polígono, (c) Encontrar a área de uma curva através da aproximação do retângulo.}
\end{figure}

\subsection{O problema da tangente}
Considere o problema de encontrar uma equação da linha tangente $t$ a uma curva com equação $y=f(x)$ em um ponto $P$ (Figura~\ref{fig:tangent-line}). Sabendo que $P$ incide sobre a tangente, basta encontrar a inclinação $m$. O problema é que precisamos de dois pontos para encontrar $m$ e só temos o ponto $P$ em $t$. Podemos encontrar uma aproximação obtendo um ponto próximo $Q$ da curva e calcular a inclinação $m_{PQ}$ da secante $PQ$. Da Figura~\ref{fig:secant-line}:
\begin{equation}\label{eq:secant-slope}
m_{PQ} =\frac{f(x)-f(a)}{x-a}
\end{equation}
Imagine $Q$ se movendo sobre a curva em direção a $P$ (Figura~\ref{fig:secant-line-tangent}). A secante se rotaciona e se aproxima da tangente como sua posição limite. Isto significa que o valor da inclinação $m_{PQ}$ se torna próximo da inclinação $m$ da tangente, ou seja $m=\lim_{Q\rightarrow P}m_{PQ}$. Como $x$ se aproxima de $a$, pela Eq.~\ref{eq:secant-slope}:\begin{equation}\label{eq:tangent-slope}
m=\lim_{x\rightarrow a}\frac{f(x)-f(a)}{x-a}
\end{equation}
\vspace{-0.6cm}\begin{figure}[!ht]
  \subfloat[\label{fig:tangent-line}]{\includegraphics{calculus/tangent-line}}%
  \subfloat[\label{fig:secant-line}]{\includegraphics{calculus/secant-line}}%
  \hspace{-0.3cm}\subfloat[\label{fig:secant-line-tangent}]{\includegraphics{calculus/secant-line-tangent}}
  \caption{Problema da tangente: (a) tangente, (b) secante, (c) secante se aproximando da tangente.}
\end{figure}

O problema da tangente é fundamental para o \emph{cálculo diferencial}, inventado apenas 2000 anos após cálculo integral. As principais ideias por trás do cálculo diferencial foram do francês Pierre Fermat (1601-1665), desenvolvidas pelo inglês John Wallis (1616-1703), Isasc Barrow (1630-1677) e Isaac Newton (1642-1727), e pelo alemão Gottfried Leibniz (1646-1716). Há uma conexão forte entre esses dois problemas.

\subsection{Velocidade}
O que significa o velocímetro indicando 60km/h ? Sabemos que se a velocidade for constante, após uma hora viajaremos 60km. E se a velocidade variar? O que significa 60km/h em um dado instante? Vamos medir a distância a cada 1 segundo num exemplo:
\begin{table}[!ht]
  \centering
  \begin{tabular}{|>{\columncolor{bookbluearea}}l|c|c|c|c|c|c|}\hline
    t=Tempo gasto(s)&0&1&2&3&4&5\\\hline
    d=Distância(m)&0&2&9&24&42&71\\\hline
  \end{tabular}
\end{table}

\noindent Qual a velocidade média quando $2\leq t \leq 4$? $$\text{velocidade média} = \frac{\text{distância}}{\text{tempo}}=\frac{42-9}{4-2}=16,5 \text{m}/\text{s}$$
Qual a velocidade média quando $2\leq t \leq 3$?
$$\text{velocidade média} = \frac{\text{distância}}{\text{tempo}}=\frac{24-9}{3-2}=15 \text{m}/\text{s}$$
Temos a sensação que a velocidade instantânea em $t=2$ não é tão diferente da velocidade média em intervalos curtos começando por $t=2$. Que tal medir a cada 0.1 segundo?
\begin{table}[!ht]
  \centering
  \begin{tabular}{|>{\columncolor{bookbluearea}}l|c|c|c|c|c|c|}\hline
    t&2,0&2,1&2,2&2,3&2,4&2,5\\\hline
    d&9,00&10,02&11,16&12,45&13,96&15,80\\\hline
  \end{tabular}
\end{table}

\noindent Fazendo a mesma divisão a cada intervalo menor, teremos:
\begin{table}[!ht]
  \centering
  \setlength\tabcolsep{0.15cm}
  \begin{tabular}{|>{\columncolor{bookbluearea}}p{1.45cm}|c|c|c|c|c|c|}\hline
    \footnotesize Intervalo de tempo    &[2-3]&[2-2,5]&[2-2,4]&[2-2,3]&[2-2,2]&[2-2,1]\\\hline
    \footnotesize Velocidade média (m/s)&15,0&13,6&12,4&11,5&10,8&10,2\\\hline
  \end{tabular}
\end{table}

\noindent Reduzindo o intervalo, a velocidade se aproxima de 10. Assim, esperamos que a velocidade instantânea em $t=2$ seja próximo de 10 m/s. Figura~\ref{fig:circle-polygon-approximation} mostra o deslocamento do carro em função do tempo. A velocidade média no intervalo [2,$t$] é $$\text{velocidade média} = \frac{\text{distância}}{\text{tempo}}=\frac{f(t)=f(2)}{t-2}$$
que é igual à secante da Figura~\ref{fig:circle-polygon-approximation}. A velocidade instantânea $v$ quando $t=2$ é o valor limite desta velocidade média quanto $t$ se aproxima de 2, $$v=\lim_{t\rightarrow 2}\frac{f(t)-f(2)}{t-2}$$ e da Eq.~\ref{eq:tangent-slope} vemos que $v$ é a inclinação da tangente da curva em $P$.
\begin{figure}[!ht]
  \includegraphics{calculus/tangent-line}
  \caption{Problema da tangente: (a) tangente, (b) secante, (c) secante se aproximando da tangente.}
\end{figure}

Podemos aplicar as mesmas técnicas de tangentes em cálculo diferencial não só em velocidades mas em todas as ciências sociais e naturais.

\subsection{O limite de uma sequência}
No 5º século A.C. o filósofo grego Zeno de Elea propôs 4 problemas, conhecidos como \emph{paradoxos de Zeno}, desafiando ideias de tempo e espaço na época. O 2º paradoxo diz respeito a uma corrida entre o herói grego Aquiles e uma tartaruga com vantagem inicial. Zeno argumentou que Aquiles nunca passaria a tartaruga: suponha que aquiles inicie na posição $a_1$ e a tartaruga em $t_1$ (Figura~\ref{fig:circle-polygon-approximation}). Quando Aquiles chega em $a_2 = t_1$, a tartaruga está em $t_2$. Quando Aquiles chega em $a_3=t_2$, a tartaruga está em $t_3$. Este processo continua indefinidamente, parecendo que a tartaruga sempre estará na frente, embora isto desafie nosso senso comum:
\begin{figure}[!ht]
  \centering
  \includegraphics{calculus/circle-polygon-approximation.pdf}
\end{figure}

\noindent Uma maneira de explicar este paradoxo é a ideia de \emph{sequência}. As posições sucessivas de Aquiles ($a_1$,$a_2$,$a_3$,$\dots$) ou da tartaruga ($t_1$,$t_2$,$t_3$,$\dots$) formam uma sequência. $\left\{1,\frac{1}{2},\frac{1}{3},\frac{1}{4},\frac{1}{5},\dots\right\}$ pode ser descrita pela seguinte expressão: $a_n=\frac{1}{n}$. Podemos visualizar esta sequência plotando seus termos em uma linha numérica (Figura~\ref{fig:circle-polygon-approximation}) ou num gráfico (Figura~\ref{fig:circle-polygon-approximation}). Observe que os termos se tornam mais próximos de 0 quando $n$ aumenta. Dizemos que o limite da sequência é 0, ou seja $$\lim_{n\rightarrow\infty}\frac{1}{n}=0$$
Em geral, a notação $\lim_{n\rightarrow\infty}a_n=L$ é usada se os termos $a_n$ se aproximam do número $L$ quando $n$ aumenta.
O conceito de limite de sequência ocorre toda vez que usamos a representação decimal de um número real. Por exemplo:
\begin{table}[!ht]
  \centering
  \setlength\tabcolsep{0.15cm}
  \begin{tabular}{|c|c|c|c|c|c|c|}\hline
    $a_1$&$a_2$&$a_3$&$a_4$&$a_5$&$a_6$&$a_7$\\\hline
    3,1&3,14&3,141&3,1415&3,14159&3,141592&3,1415926\\\hline
  \end{tabular}
\end{table}

\noindent Então os termos desta sequência são aproximações racionais de $\pi$, ou seja, $\lim_{n\rightarrow \infty}a_n=\pi$. Voltando ao paradoxo de Zeno. As posições de Aquiles e da tartaruga formam sequências $\{a_n\}$ e $\{t_n\}$, onde $a_n<t_n$, para todo $n$. Pode-se mostrar que ambas as sequências têm o mesmo limite: $\lim_{n\rightarrow\infty}a_n=p=\lim_{n\rightarrow\infty}t_n$. É neste ponto $p$ onde Aquiles supera a tartaruga.

\subsection{Soma de uma série}
a
\subsection{Sumário}
a

\section{Funções e limites}
\section{Derivadas}
\section{Integrais}
\section{Funções inversas}
\section{Técnicas de integração}
\section{Aplicações de integração}
\section{Equações diferenciais}
\section{Equações paramétricas e coordenadas polares}
\section{Sequências infinitas e séries}
