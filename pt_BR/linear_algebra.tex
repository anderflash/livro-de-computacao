% !TeX root = main.tex
\part{Álgebra Linear}
\section{Equações Lineares e Álgebra Linear}
\subsection{Sistemas de equações lineares}
Uma \textbf{equação linear} nas variáveis $x_1,\dots,x_n$ é uma equação que pode ser escrita na forma \begin{equation}\label{eq:linear-equation}
a_1x_1+a_2x_2+\dots+a_nx_n = b
\end{equation}
onde $b$ e os coeficientes $a_1,\dots,a_n$ são números reais ou complexos, geralmente conhecidos \textit{a priori}, e $n\in\mathds{N}^+$. As equações $4x_1-5x_2+2 = x_1$ e $x_2 = 2(\sqrt{6}-x_1)+x_3$ são lineares pois podem ser escritos como na Equação~\ref{eq:linear-equation}: $3x_1-5x_2=-2$ e $2x_1+x_2-x_3=2\sqrt{6}$. As equações $4x_1-5x_2=x_1x_2$ e $x_2=2\sqrt{x_1}-6$ não são lineares devido à presença de $x_1x_2$ e $\sqrt{x_1}$.

Um \textbf{sistema de equações lineares} (um ou \textbf{sistema linear}) é uma coleção de uma ou mais equações lineares envolvendo as mesmas variáveis $x_1,\dots,x_n$. Um exemplo é \begin{equation}
\begin{aligned}
2x_1 - x_2 &+ 1.5x_3& = 8\\
x_1 &- 4x_3&   = -7
\end{aligned}
\end{equation}

\subsection{Redução de linha e forma escalonada}
\subsection{Equações vetoriais}
\subsection{A equação de matriz Ax=b}
\subsection{Conjunto solução de sistemas lineares}
\subsection{Aplicações de sistemas lineares}
\subsection{Independência linear}
\subsection{Introdução a transformações lineares}
\subsection{Método de Gauss}

\section{Determinantes}
\section{Espaços Vetoriais Euclidianos}
\section{Espaços Vetoriais Arbitrários}
\section{Autovalores e Autovetores}
\section{Espaços com Produto Interno}
\section{Diagonalização e Formas Quadráticas}
\section{Transformações Lineares}
\section{Métodos Numéricos}
\section{Aplicações de Álgebra Linear}
